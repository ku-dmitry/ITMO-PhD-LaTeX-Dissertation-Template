\chapter{Оформление различных элементов}\label{ch:ch1}

Цитируем \cite{confbib1}

Вот так пишется нумерованная формула:
\begin{equation}
  \label{eq:equation1}
  e = \lim_{n \to \infty} \left( 1+\frac{1}{n} \right) ^n
\end{equation}

Нумерованных формул может быть несколько:
\begin{equation}
  \label{eq:equation2}
  \lim_{n \to \infty} \sum_{k=1}^n \frac{1}{k^2} = \frac{\pi^2}{6}
\end{equation}

Впоследствии на формулы~\eqref{eq:equation1} и~\eqref{eq:equation2} можно ссылаться.

Сделать так, чтобы номер формулы стоял напротив средней строки, можно,
используя окружение \verb|multlined| (пакет \verb|mathtools|) вместо
\verb|multline| внутри окружения \verb|equation|. Вот так:
\begin{equation} % \tag{S} % tag - вписывает свой текст
  \label{eq:equation3}
    \begin{multlined}
        1+ 2+3+4+5+6+7+\dots + \\
        + 50+51+52+53+54+55+56+57 + \dots + \\
        + 96+97+98+99+100=5050
    \end{multlined}
\end{equation}

Используя команду \verb|\labelcref| из пакета \verb|cleveref|, можно
красиво ссылаться сразу на несколько формул
(\labelcref{eq:equation1, eq:equation3, eq:equation2}), даже перепутав
порядок ссылок \verb|(\labelcref{eq:equation1, eq:equation3, eq:equation2})|.
