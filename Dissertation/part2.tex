\chapter{Длинное название главы, в которой мы смотрим на~примеры того, как будут верстаться изображения и~списки}\label{ch:ch2}

\section{Одиночное изображение}\label{sec:ch2/sec1}

\begin{figure}[ht]
  \centerfloat{
    \includegraphics[scale=0.27]{latex}
  }
  \caption{TeX.}\label{fig:latex}
\end{figure}

Для выравнивания изображения по-центру используется команда \verb+\centerfloat+, которая является во
многом улучшенной версией встроенной команды \verb+\centering+.

\section{Длинное название параграфа, в котором мы узнаём как сделать две картинки с~общим номером и названием}\label{sec:ch2/sect2}

А это две картинки под общим номером и названием:
\begin{figure}[ht]
  \begin{minipage}[b][][b]{0.49\linewidth}\centering
    \includegraphics[width=0.5\linewidth]{knuth1} \\ а)
  \end{minipage}
  \hfill
  \begin{minipage}[b][][b]{0.49\linewidth}\centering
    \includegraphics[width=0.5\linewidth]{knuth2} \\ б)
  \end{minipage}
  \caption{Очень длинная подпись к изображению,
      на котором представлены две фотографии Дональда Кнута}
  \label{fig:knuth}
\end{figure}

Те~же~две картинки под~общим номером и~названием,
но с автоматизированной нумерацией подрисунков:
\begin{figure}[ht]
    \centerfloat{
        \hfill
        \subbottom[List-of-Figures entry][Первый подрисунок\label{fig:knuth_2-1}]{%
            \includegraphics[width=0.25\linewidth]{knuth1}}
        \hfill
        \subbottom[\label{fig:knuth_2-2}]{%
            \includegraphics[width=0.25\linewidth]{knuth2}}
        \hfill
        \subbottom[Третий подрисунок]{%
            \includegraphics[width=0.3\linewidth]{example-image-c}}
        \hfill
    }
    \legend{Подрисуночный текст, описывающий обозначения, например. Согласно
    ГОСТ 2.105, пункт 4.3.1, располагается перед наименованием рисунка.}
    \caption[Этот текст попадает в названия рисунков в списке рисунков]{Очень
    длинная подпись к второму изображению, на~котором представлены две
    фотографии Дональда Кнута}\label{fig:knuth_2}
\end{figure}

На рисунке~\ref{fig:knuth_2-1} показан Дональд Кнут без головного убора.
На рисунке~\ref{fig:knuth_2}\subcaptionref*{fig:knuth_2-2}
показан Дональд Кнут в головном уборе.
