%&preformat-disser
\RequirePackage[l2tabu,orthodox]{nag} % Раскомментировав, можно в логе получать рекомендации относительно правильного использования пакетов и предупреждения об устаревших и нерекомендуемых пакетах
% Формат А4, 14pt (ГОСТ Р 7.0.11-2011, 5.3.6)
\documentclass[a4paper,11pt,oneside,openany]{memoir}

\input{common/setup}            % общие настройки шаблона
\input{common/packages}         % Пакеты общие для диссертации и автореферата
\synopsisfalse                      % Этот документ --- не автореферат
\input{Dissertation/dispackages}    % Пакеты для диссертации
\input{Dissertation/userpackages}   % Пакеты для специфических пользовательских задач

\input{Dissertation/setup}      % Упрощённые настройки шаблона

\input{common/newnames}         % Новые переменные, для всего проекта

\input{common/data}             % Основные сведения
\input{common/fonts}            % Определение шрифтов (частичное)
\input{common/styles}           % Стили общие для диссертации и автореферата
\input{Dissertation/disstyles}  % Стили для диссертации
\input{Dissertation/userstyles} % Стили для специфических пользовательских задач

%%% Библиография. Выбор движка для реализации %%%
\ifnumequal{\value{bibliosel}}{1}{%
    \input{biblio/predefined}   % Встроенная реализация с загрузкой файла через движок bibtex8
}{
    \input{biblio/biblatex}     % Реализация пакетом biblatex через движок biber
}

% Вывести информацию о выбранных опциях в лог сборки
%\typeout{Selected options:}
%\typeout{Draft mode: \arabic{draft}}
%\typeout{Font: \arabic{fontfamily}}
%\typeout{AltFont: \arabic{usealtfont}}
%\typeout{Bibliography backend: \arabic{bibliosel}}
%\typeout{Precompile images: \arabic{imgprecompile}}
% Вывести информацию о версиях используемых библиотек в лог сборки
\listfiles

%%% Управление компиляцией отдельных частей диссертации %%%
% Необходимо сначала иметь полностью скомпилированный документ, чтобы все
% промежуточные файлы были в наличии
% Затем, для вывода отдельных частей можно воспользоваться командой \includeonly
% Ниже примеры использования команды:
%
%\includeonly{Dissertation/part2}
%\includeonly{Dissertation/contents,Dissertation/appendix,Dissertation/conclusion}
%
% Если все команды закомментированы, то документ будет выведен в PDF файл полностью

\begin{document}

\input{common/renames}                 % Переопределение именований

%%% Структура диссертации (ГОСТ Р 7.0.11-2011, 4)
\setcounter{page}{5}
\include{Dissertation/contents}        % Оглавление
\include{Dissertation/ref}
\include{Dissertation/syn}

\include{Dissertation/acronyms}        % Список сокращений
\chapter*{Термины (необязательно)} % Заголовок
\addcontentsline{toc}{chapter}{Термины (необязательно)}  % Добавляем его в оглавление
\noindent
%\begin{longtabu} to \dimexpr \textwidth-5\tabcolsep {r X}
\begin{longtabu} to \textwidth {r X}
% Жирное начертание для математических символов может иметь
% дополнительный смысл, поэтому они приводятся как в тексте
% диссертации
\(\begin{rcases}
a_n\\
b_n
\end{rcases}\)  &
\begin{minipage}{\linewidth}
коэффициенты разложения Ми в дальнем поле соответствующие
электрическим и магнитным мультиполям
\end{minipage}
\\
\({\boldsymbol{\hat{\mathrm e}}}\) & единичный вектор \\
\(E_0\) & амплитуда падающего поля\\

\end{longtabu}
\addtocounter{table}{-1}% Нужно откатить на единицу счетчик номеров таблиц, так как предыдующая таблица сделана для удобства представления информации по ГОСТ
         % Список терминов
\include{Dissertation/shortcuts}       % Список условных обозначений
\chapter*{Введение}                         % Заголовок
\addcontentsline{toc}{chapter}{Введение}
\noindent Текст, первый абзац без отступа. Во введении в диссертацию нужно отразить актуальность темы исследования и причины выбора темы, цели и задачи диссертации, проблему, подлежащую исследованию, методы, использованные в диссертации, использованную исходную информацию, теоретическую и практическую новизну исследования, обзор апробации результатов исследования (включая конференции и семинары, опубликованные статьи и т.д.).
Следующий параграф с отступом.
    % Введение
\chapter{Оформление различных элементов}\label{ch:ch1}

Цитируем \cite{confbib1}

Вот так пишется нумерованная формула:
\begin{equation}
  \label{eq:equation1}
  e = \lim_{n \to \infty} \left( 1+\frac{1}{n} \right) ^n
\end{equation}

Нумерованных формул может быть несколько:
\begin{equation}
  \label{eq:equation2}
  \lim_{n \to \infty} \sum_{k=1}^n \frac{1}{k^2} = \frac{\pi^2}{6}
\end{equation}

Впоследствии на формулы~\eqref{eq:equation1} и~\eqref{eq:equation2} можно ссылаться.

Сделать так, чтобы номер формулы стоял напротив средней строки, можно,
используя окружение \verb|multlined| (пакет \verb|mathtools|) вместо
\verb|multline| внутри окружения \verb|equation|. Вот так:
\begin{equation} % \tag{S} % tag - вписывает свой текст
  \label{eq:equation3}
    \begin{multlined}
        1+ 2+3+4+5+6+7+\dots + \\
        + 50+51+52+53+54+55+56+57 + \dots + \\
        + 96+97+98+99+100=5050
    \end{multlined}
\end{equation}

Используя команду \verb|\labelcref| из пакета \verb|cleveref|, можно
красиво ссылаться сразу на несколько формул
(\labelcref{eq:equation1, eq:equation3, eq:equation2}), даже перепутав
порядок ссылок \verb|(\labelcref{eq:equation1, eq:equation3, eq:equation2})|.
           % Глава 1
\chapter{Длинное название главы, в которой мы смотрим на~примеры того, как будут верстаться изображения и~списки}\label{ch:ch2}

\section{Одиночное изображение}\label{sec:ch2/sec1}

\begin{figure}[ht]
  \centerfloat{
    \includegraphics[scale=0.27]{latex}
  }
  \caption{TeX.}\label{fig:latex}
\end{figure}

Для выравнивания изображения по-центру используется команда \verb+\centerfloat+, которая является во
многом улучшенной версией встроенной команды \verb+\centering+.

\section{Длинное название параграфа, в котором мы узнаём как сделать две картинки с~общим номером и названием}\label{sec:ch2/sect2}

А это две картинки под общим номером и названием:
\begin{figure}[ht]
  \begin{minipage}[b][][b]{0.49\linewidth}\centering
    \includegraphics[width=0.5\linewidth]{knuth1} \\ а)
  \end{minipage}
  \hfill
  \begin{minipage}[b][][b]{0.49\linewidth}\centering
    \includegraphics[width=0.5\linewidth]{knuth2} \\ б)
  \end{minipage}
  \caption{Очень длинная подпись к изображению,
      на котором представлены две фотографии Дональда Кнута}
  \label{fig:knuth}
\end{figure}

Те~же~две картинки под~общим номером и~названием,
но с автоматизированной нумерацией подрисунков:
\begin{figure}[ht]
    \centerfloat{
        \hfill
        \subbottom[List-of-Figures entry][Первый подрисунок\label{fig:knuth_2-1}]{%
            \includegraphics[width=0.25\linewidth]{knuth1}}
        \hfill
        \subbottom[\label{fig:knuth_2-2}]{%
            \includegraphics[width=0.25\linewidth]{knuth2}}
        \hfill
        \subbottom[Третий подрисунок]{%
            \includegraphics[width=0.3\linewidth]{example-image-c}}
        \hfill
    }
    \legend{Подрисуночный текст, описывающий обозначения, например. Согласно
    ГОСТ 2.105, пункт 4.3.1, располагается перед наименованием рисунка.}
    \caption[Этот текст попадает в названия рисунков в списке рисунков]{Очень
    длинная подпись к второму изображению, на~котором представлены две
    фотографии Дональда Кнута}\label{fig:knuth_2}
\end{figure}

На рисунке~\ref{fig:knuth_2-1} показан Дональд Кнут без головного убора.
На рисунке~\ref{fig:knuth_2}\subcaptionref*{fig:knuth_2-2}
показан Дональд Кнут в головном уборе.
           % Глава 2
\chapter{Вёрстка таблиц}\label{ch:ch3}

\section{Таблица обыкновенная}\label{sec:ch3/sect1}

Так размещается таблица:

\begin{table} [htbp]
  \centering
  \changecaptionwidth\captionwidth{15cm}
  \caption{Название таблицы}\label{tab:Ts0Sib}%
  \begin{tabular}{| p{3cm} || p{3cm} | p{3cm} | p{4cm}l |}
  \hline
  \hline
  Месяц   & \centering \(T_{min}\), К & \centering \(T_{max}\), К &\centering  \((T_{max} - T_{min})\), К & \\
  \hline
  Декабрь &\centering  253.575   &\centering  257.778    &\centering      4.203  &   \\
  Январь  &\centering  262.431   &\centering  263.214    &\centering      0.783  &   \\
  Февраль &\centering  261.184   &\centering  260.381    &\centering     \(-\)0.803  &   \\
  \hline
  \hline
  \end{tabular}
\end{table}

\begin{table} [htbp]% Пример записи таблицы с номером, но без отображаемого наименования
    \centering
    \parbox{9cm}{% чтобы лучше смотрелось, подбирается самостоятельно
        \captiondelim{}% должен стоять до самого пустого caption
        \caption{}%
        \label{tab:test1}%
        \begin{SingleSpace}
            \begin{tabular}{| c | c | c | c |}
                \hline
                Оконная функция & \({2N}\)& \({4N}\)& \({8N}\)\\ \hline
                Прямоугольное   & 8.72  & 8.77  & 8.77  \\ \hline
                Ханна           & 7.96  & 7.93  & 7.93  \\ \hline
                Хэмминга        & 8.72  & 8.77  & 8.77  \\ \hline
                Блэкмана        & 8.72  & 8.77  & 8.77  \\ \hline
            \end{tabular}%
        \end{SingleSpace}
    }
\end{table}

Таблица~\ref{tab:test2} "--- пример таблицы, оформленной в~классическом книжном
варианте или~очень близко к~нему. \mbox{ГОСТу} по~сути не~противоречит. Можно
ещё~улучшить представление, с~помощью пакета \verb|siunitx| или~подобного.

\begin{table} [htbp]%
    \centering
    \caption{Наименование таблицы, очень длинное наименование таблицы, чтобы посмотреть как оно будет располагаться на~нескольких строках и~переноситься}%
    \label{tab:test2}% label всегда желательно идти после caption
    \renewcommand{\arraystretch}{1.5}%% Увеличение расстояния между рядами, для улучшения восприятия.
    \begin{SingleSpace}
        \begin{tabular}{@{}@{\extracolsep{20pt}}llll@{}} %Вертикальные полосы не используются принципиально, как и лишние горизонтальные (допускается по ГОСТ 2.105 пункт 4.4.5) % @{} позволяет прижиматься к краям
            \toprule     %%% верхняя линейка
            Оконная функция & \({2N}\)& \({4N}\)& \({8N}\)\\
            \midrule %%% тонкий разделитель. Отделяет названия столбцов. Обязателен по ГОСТ 2.105 пункт 4.4.5
            Прямоугольное   & 8.72  & 8.77  & 8.77  \\
            Ханна           & 7.96  & 7.93  & 7.93  \\
            Хэмминга        & 8.72  & 8.77  & 8.77  \\
            Блэкмана        & 8.72  & 8.77  & 8.77  \\
            \bottomrule %%% нижняя линейка
        \end{tabular}%
    \end{SingleSpace}
\end{table}

\section{Таблица с многострочными ячейками и примечанием}

В таблице~\ref{tab:makecell} приведён пример использования команды
\verb+\multicolumn+ для объединения горизонтальных ячеек таблицы,
и команд пакета \textit{makecell} для добавления разрыва строки внутри ячеек.

\begin{table} [htbp]
	\centering
	\caption{Пример использования функций пакета \textit{makecell}.}%
	\label{tab:makecell}%
	\begin{tabular}{| c | c | c | c |}
	  \hline
	  Колонка 1                                    & Колонка 2        & \thead{Название колонки 3, \\ не помещающееся в одну строку} & Колонка 4 \\ \hline
	  \multicolumn{4}{|c|}{Выравнивание по центру}                                                                                               \\ \hline
	  \multicolumn{2}{|r|}{\makecell{Выравнивание к \\ правому краю}} & \multicolumn{2}{|l|}{Выравнивание к левому краю}                         \\ \hline
	  \makecell{В этой ячейке \\ много информации} & 8.72             & 8.55                                                         & 8.44      \\ \cline{3-4}
	  А в этой мало                                & 8.22             & \multicolumn{2}{|c|}{5}                                                  \\ \hline
	\end{tabular}%
\end{table}


\section{Параграф "--- два}\label{sec:ch3/sect2}

Некоторый текст.

\section{Параграф с подпараграфами}\label{sec:ch3/sect3}

\subsection{Подпараграф "--- один}\label{subsec:ch3/sect3/sub1}

Некоторый текст.

\subsection{Подпараграф "--- два}\label{subsec:ch3/sect3/sub2}

Некоторый текст.

\clearpage           % Глава 3
\include{Dissertation/lists}           % Список таблиц и изображений
\include{Dissertation/references}
\chapter*{Благодарности}
\addcontentsline{toc}{chapter}{Благодарности}
\noindent Здесь Вы можете поблагодарить своего научного руководителя, коллег, членов семьи и т.д. за поддержку.
Не забудьте также упомянуть все финансирующие организации, которые оказывали поддержку Вашему исследованию.

	

\chapter*{Приложение 1}
\addcontentsline{toc}{chapter}{Приложение 1}


	
\chapter*{Приложение 2}
\addcontentsline{toc}{chapter}{Приложение 2}

	
\chapter*{Публикации автора по теме диссертации}
\addcontentsline{toc}{chapter}{Публикации автора по теме диссертации}
Список публикаций автора, на основании которых подготовлена диссертация:

\begin{enumerate}[label=\Roman*., leftmargin=\parindent, align=left]
	\item Статья 1
	\item Статья 2
	\item Статья 3
	\item
\end{enumerate}
	    % Список публикаций
%%% Настройки для приложений
\appendix
% Оформление заголовков приложений ближе к ГОСТ:
\setlength{\midchapskip}{20pt}
\renewcommand*{\afterchapternum}{\par\nobreak\vskip \midchapskip}

%\include{Dissertation/appendix}        % Приложения
\chapter*{Резюме}
\addcontentsline{toc}{chapter}{Резюме}
	Имя:

	Дата рождения:

	Место рождения:

	Гражданство:

Контактные данные:

	E-mail:

Образование:

201 (год) – 201 (год) Университет ИТМО – к.т.н.

201 (год) – 201 (год) Университет ИТМО Магистр

201 (год) – 201 (год) Университет ИТМО Бакалавр

Владение языками

Английский Свободно

Профессиональное трудоустройство

201 (год) –

\end{document}
