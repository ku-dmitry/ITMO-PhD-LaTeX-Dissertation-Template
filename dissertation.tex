%&preformat-disser
\RequirePackage[l2tabu,orthodox]{nag} % Раскомментировав, можно в логе получать рекомендации относительно правильного использования пакетов и предупреждения об устаревших и нерекомендуемых пакетах
% Формат А4, 14pt (ГОСТ Р 7.0.11-2011, 5.3.6)
\documentclass[a4paper,11pt,oneside,openany]{memoir}

\input{common/setup}            % общие настройки шаблона
\input{common/packages}         % Пакеты общие для диссертации и автореферата
\synopsisfalse                      % Этот документ --- не автореферат
\input{Dissertation/dispackages}    % Пакеты для диссертации
\input{Dissertation/userpackages}   % Пакеты для специфических пользовательских задач

\input{Dissertation/setup}      % Упрощённые настройки шаблона

\input{common/newnames}         % Новые переменные, для всего проекта

\input{common/data}             % Основные сведения
\input{common/fonts}            % Определение шрифтов (частичное)
%%% Шаблон %%%
\DeclareRobustCommand{\todo}{\textcolor{red}}       % решаем проблему превращения названия цвета в результате \MakeUppercase, http://tex.stackexchange.com/a/187930, \DeclareRobustCommand protects \todo from expanding inside \MakeUppercase
\AtBeginDocument{%
    \setlength{\parindent}{1.25em}                   % Абзацный отступ. Должен быть одинаковым по всему тексту и равен пяти знакам (ГОСТ Р 7.0.11-2011, 5.3.7).
}

%%% Подписи %%%
\setlength{\abovecaptionskip}{0pt}   % Отбивка над подписью
\setlength{\belowcaptionskip}{0pt}   % Отбивка под подписью
\captionwidth{\linewidth}
\normalcaptionwidth

%%% Таблицы %%%
\ifnumequal{\value{tabcap}}{0}{%
    \newcommand{\tabcapalign}{\raggedright}  % по левому краю страницы или аналога parbox
    \renewcommand{\tablabelsep}{~\cyrdash\ } % тире как разделитель идентификатора с номером от наименования
    \newcommand{\tabtitalign}{}
}{%
    \ifnumequal{\value{tablaba}}{0}{%
        \newcommand{\tabcapalign}{\raggedright}  % по левому краю страницы или аналога parbox
    }{}

    \ifnumequal{\value{tablaba}}{1}{%
        \newcommand{\tabcapalign}{\centering}    % по центру страницы или аналога parbox
    }{}

    \ifnumequal{\value{tablaba}}{2}{%
        \newcommand{\tabcapalign}{\raggedleft}   % по правому краю страницы или аналога parbox
    }{}

    \ifnumequal{\value{tabtita}}{0}{%
        \newcommand{\tabtitalign}{\par\raggedright}  % по левому краю страницы или аналога parbox
    }{}

    \ifnumequal{\value{tabtita}}{1}{%
        \newcommand{\tabtitalign}{\par\centering}    % по центру страницы или аналога parbox
    }{}

    \ifnumequal{\value{tabtita}}{2}{%
        \newcommand{\tabtitalign}{\par\raggedleft}   % по правому краю страницы или аналога parbox
    }{}
}

\precaption{\tabcapalign} % всегда идет перед подписью или \legend
\captionnamefont{\normalfont\normalsize} % Шрифт надписи «Таблица #»; также определяет шрифт у \legend
\captiondelim{\tablabelsep} % разделитель идентификатора с номером от наименования
\captionstyle[\tabtitalign]{\tabtitalign}
\captiontitlefont{\normalfont\normalsize} % Шрифт с текстом подписи

%%% Рисунки %%%
\setfloatadjustment{figure}{%
    \setlength{\abovecaptionskip}{0pt}   % Отбивка над подписью
    \setlength{\belowcaptionskip}{0pt}   % Отбивка под подписью
    \precaption{} % всегда идет перед подписью или \legend
    \captionnamefont{\normalfont\normalsize} % Шрифт надписи «Рисунок #»; также определяет шрифт у \legend
    \captiondelim{\figlabelsep} % разделитель идентификатора с номером от наименования
    \captionstyle[\centering]{\centering} % Центрирование подписей, заданных командой \caption и \legend
    \captiontitlefont{\normalfont\normalsize} % Шрифт с текстом подписи
    \postcaption{} % всегда идет после подписи или \legend, и с новой строки
}

%%% Подписи подрисунков %%%
\newsubfloat{figure} % Включает возможность использовать подрисунки у окружений figure
\renewcommand{\thesubfigure}{\asbuk{subfigure}}           % Буквенные номера подрисунков
\subcaptionsize{\normalsize} % Шрифт подписи названий подрисунков (не отличается от основного)
\subcaptionlabelfont{\normalfont}
\subcaptionfont{\!\!) \normalfont} % Вот так тут добавили скобку после буквы.
\subcaptionstyle{\centering}
%\subcaptionsize{\fontsize{12pt}{13pt}\selectfont} % объявляем шрифт 12pt для использования в подписях, тут же надо интерлиньяж объявлять, если не наследуется

%%% Настройки гиперссылок %%%
\ifluatex
    \hypersetup{
        unicode,                % Unicode encoded PDF strings
    }
\fi

\hypersetup{
    linktocpage=true,           % ссылки с номера страницы в оглавлении, списке таблиц и списке рисунков
%    linktoc=all,                % both the section and page part are links
%    pdfpagelabels=false,        % set PDF page labels (true|false)
    plainpages=false,           % Forces page anchors to be named by the Arabic form  of the page number, rather than the formatted form
    colorlinks,                 % ссылки отображаются раскрашенным текстом, а не раскрашенным прямоугольником, вокруг текста
    linkcolor={linkcolor},      % цвет ссылок типа ref, eqref и подобных
    citecolor={citecolor},      % цвет ссылок-цитат
    urlcolor={urlcolor},        % цвет гиперссылок
%    hidelinks,                  % Hide links (removing color and border)
    pdftitle={\thesisTitle},    % Заголовок
    pdfauthor={\thesisAuthor},  % Автор
    pdfsubject={\thesisSpecialtyNumber\ \thesisSpecialtyTitle},      % Тема
%    pdfcreator={Создатель},     % Создатель, Приложение
%    pdfproducer={Производитель},% Производитель, Производитель PDF
    pdfkeywords={\keywords},    % Ключевые слова
    pdflang={ru},
}
\ifnumequal{\value{draft}}{1}{% Черновик
    \hypersetup{
        draft,
    }
}{}

%%% Списки %%%
% Используем короткое тире (endash) для ненумерованных списков (ГОСТ 2.105-95, пункт 4.1.7, требует дефиса, но так лучше смотрится)
\renewcommand{\labelitemi}{\normalfont\bfseries{--}}

% Перечисление строчными буквами латинского алфавита (ГОСТ 2.105-95, 4.1.7)
%\renewcommand{\theenumi}{\alph{enumi}}
%\renewcommand{\labelenumi}{\theenumi)}

% Перечисление строчными буквами русского алфавита (ГОСТ 2.105-95, 4.1.7)
\makeatletter
\AddEnumerateCounter{\asbuk}{\russian@alph}{щ}      % Управляем списками/перечислениями через пакет enumitem, а он 'не знает' про asbuk, потому 'учим' его
\makeatother
%\renewcommand{\theenumi}{\asbuk{enumi}} %первый уровень нумерации
%\renewcommand{\labelenumi}{\theenumi)} %первый уровень нумерации
\renewcommand{\theenumii}{\asbuk{enumii}} %второй уровень нумерации
\renewcommand{\labelenumii}{\theenumii)} %второй уровень нумерации
\renewcommand{\theenumiii}{\arabic{enumiii}} %третий уровень нумерации
\renewcommand{\labelenumiii}{\theenumiii)} %третий уровень нумерации

\setlist{nosep,%                                    % Единый стиль для всех списков (пакет enumitem), без дополнительных интервалов.
    labelindent=\parindent,leftmargin=*%            % Каждый пункт, подпункт и перечисление записывают с абзацного отступа (ГОСТ 2.105-95, 4.1.8)
}
           % Стили общие для диссертации и автореферата
%%% Переопределение именований, если иначе не сработает %%%
%\gappto\captionsrussian{
%    \renewcommand{\chaptername}{Глава}
%    \renewcommand{\appendixname}{Приложение} % (ГОСТ Р 7.0.11-2011, 5.7)
%}

%%% Изображения %%%
\graphicspath{{images/}{Dissertation/images/}}         % Пути к изображениям

%%% Интервалы %%%
%% По ГОСТ Р 7.0.11-2011, пункту 5.3.6 требуется полуторный интервал
%% Реализация средствами класса (на основе setspace) ближе к типографской классике.
%% И правит сразу и в таблицах (если со звёздочкой)
%\DoubleSpacing*     % Двойной интервал
%\OneSpacing*    % Полуторный интервал
\setSpacing{1}   % Одинарный интервал, подобный Ворду (возможно, стоит включать вместе с предыдущей строкой)

%%% Макет страницы %%%
% Выставляем значения полей (ГОСТ 7.0.11-2011, 5.3.7)
\geometry{a4paper, top=2cm, bottom=2cm, left=2.5cm, right=2.5cm, nofoot, nomarginpar} %, heightrounded, showframe
\setlength{\topskip}{0pt}   %размер дополнительного верхнего поля
\setlength{\footskip}{12.3pt} % снимет warning, согласно https://tex.stackexchange.com/a/334346

%%% Выравнивание и переносы %%%
%% http://tex.stackexchange.com/questions/241343/what-is-the-meaning-of-fussy-sloppy-emergencystretch-tolerance-hbadness
%% http://www.latex-community.org/forum/viewtopic.php?p=70342#p70342
\tolerance 1414
\hbadness 1414
\emergencystretch 1.5em % В случае проблем регулировать в первую очередь
\hfuzz 0.3pt
\vfuzz \hfuzz
%\raggedbottom
%\sloppy                 % Избавляемся от переполнений
\clubpenalty=10000      % Запрещаем разрыв страницы после первой строки абзаца
\widowpenalty=10000     % Запрещаем разрыв страницы после последней строки абзаца
\brokenpenalty=4991     % Ограничение на разрыв страницы, если строка заканчивается переносом

%%% Блок управления параметрами для выравнивания заголовков в тексте %%%
\newlength{\otstuplen}
\setlength{\otstuplen}{\theotstup\parindent}
\ifnumequal{\value{headingalign}}{1}{% выравнивание заголовков в тексте
    \newcommand{\hdngalign}{\centering}                % по центру
    \newcommand{\hdngaligni}{}% по центру
    \setlength{\otstuplen}{0pt}
}{%
    \newcommand{\hdngalign}{}                 % по левому краю
    \newcommand{\hdngaligni}{\hspace{\otstuplen}}      % по левому краю
} % В обоих случаях вроде бы без переноса, как и надо (ГОСТ Р 7.0.11-2011, 5.3.5)

%%% Оглавление %%%
\renewcommand{\cftchapterdotsep}{\cftdotsep}                % отбивка точками до номера страницы начала главы/раздела

%% Переносить слова в заголовке не допускается (ГОСТ Р 7.0.11-2011, 5.3.5). Заголовки в оглавлении должны точно повторять заголовки в тексте (ГОСТ Р 7.0.11-2011, 5.2.3). Прямого указания на запрет переносов в оглавлении нет, но по той же логике невнесения искажений в смысл, лучше в оглавлении не переносить:
\setrmarg{2.55em plus1fil}                             %To have the (sectional) titles in the ToC, etc., typeset ragged right with no hyphenation
\renewcommand{\cftchapterpagefont}{\normalfont}        % нежирные номера страниц у глав в оглавлении
\renewcommand{\cftchapterleader}{\cftdotfill{\cftchapterdotsep}}% нежирные точки до номеров страниц у глав в оглавлении
%\renewcommand{\cftchapterfont}{}                       % нежирные названия глав в оглавлении

\ifnumgreater{\value{headingdelim}}{0}{%
    \renewcommand\cftchapteraftersnum{.\space}       % добавляет точку с пробелом после номера раздела в оглавлении
}{}
\ifnumgreater{\value{headingdelim}}{1}{%
    \renewcommand\cftsectionaftersnum{.\space}       % добавляет точку с пробелом после номера подраздела в оглавлении
    \renewcommand\cftsubsectionaftersnum{.\space}    % добавляет точку с пробелом после номера подподраздела в оглавлении
    \renewcommand\cftsubsubsectionaftersnum{.\space} % добавляет точку с пробелом после номера подподподраздела в оглавлении
    \AtBeginDocument{% без этого polyglossia сама всё переопределяет
        \setsecnumformat{\csname the#1\endcsname.\space}
    }
}{%
    \AtBeginDocument{% без этого polyglossia сама всё переопределяет
        \setsecnumformat{\csname the#1\endcsname\quad}
    }
}

\renewcommand*{\cftappendixname}{\appendixname\space} % Слово Приложение в оглавлении

%%% Колонтитулы %%%
% Порядковый номер страницы печатают на середине верхнего поля страницы (ГОСТ Р 7.0.11-2011, 5.3.8)
\makeevenhead{plain}{}{}{}
\makeoddhead{plain}{}{}{}
\makeevenfoot{plain}{}{\thepage}{}
\makeoddfoot{plain}{}{\thepage}{}
\pagestyle{plain}

%%% добавить Стр. над номерами страниц в оглавлении
%%% http://tex.stackexchange.com/a/306950
%\newif\ifendTOC

%\newcommand*{\tocheader}{
%\ifnumequal{\value{pgnum}}{1}{%
%    \ifendTOC\else\hbox to \linewidth%
%      {\noindent{}~\hfill{Стр.}}\par%
%      \ifnumless{\value{page}}{3}{}{%
%        \vspace{0.5\onelineskip}
%      }
%      \afterpage{\tocheader}
%    \fi%
%}{}%
%}%

%%% Оформление заголовков глав, разделов, подразделов %%%
%% Работа должна быть выполнена ... размером шрифта 12-14 пунктов (ГОСТ Р 7.0.11-2011, 5.3.8). То есть не должно быть надписей шрифтом более 14. Так и поставим.
%% Эти установки будут давать одинаковый результат независимо от выбора базовым шрифтом 12 пт или 14 пт
\newcommand{\basegostsectionfont}{\fontsize{14pt}{16pt}\selectfont\bfseries}

\makechapterstyle{thesisgost}{%
    \chapterstyle{default}
    \setlength{\beforechapskip}{0pt}
    \setlength{\midchapskip}{0pt}
    \setlength{\afterchapskip}{\theintvl\curtextsize}
    \renewcommand*{\chapnamefont}{\basegostsectionfont}
    \renewcommand*{\chapnumfont}{\basegostsectionfont}
    \renewcommand*{\chaptitlefont}{\basegostsectionfont}
    \renewcommand*{\chapterheadstart}{}
    \ifnumgreater{\value{headingdelim}}{0}{%
        \renewcommand*{\afterchapternum}{.\space}   % добавляет точку с пробелом после номера раздела
    }{%
        \renewcommand*{\afterchapternum}{\quad}     % добавляет \quad после номера раздела
    }
    \renewcommand*{\printchapternum}{\hdngaligni\hdngalign\chapnumfont \thechapter}
    \renewcommand*{\printchaptername}{}
    \renewcommand*{\printchapternonum}{\hdngaligni\hdngalign}
}

\makeatletter
\makechapterstyle{thesisgostchapname}{%
    \chapterstyle{thesisgost}
    \renewcommand*{\printchapternum}{\chapnumfont \thechapter}
    \renewcommand*{\printchaptername}{\hdngaligni\hdngalign\chapnamefont \@chapapp} %
}
\makeatother

\chapterstyle{thesisgost}

\setsecheadstyle{\basegostsectionfont\hdngalign}
\setsecindent{\otstuplen}

\setsubsecheadstyle{\basegostsectionfont\hdngalign}
\setsubsecindent{\otstuplen}

\setsubsubsecheadstyle{\basegostsectionfont\hdngalign}
\setsubsubsecindent{\otstuplen}

\sethangfrom{\noindent #1} %все заголовки подразделов центрируются с учетом номера, как block

%\ifnumequal{\value{chapstyle}}{1}{%
%    \chapterstyle{thesisgostchapname}
%    \renewcommand*{\cftchaptername}{\chaptername\space} % будет вписано слово Глава перед каждым номером раздела в оглавлении
%}{}%

%%% Интервалы между заголовками
\setbeforesecskip{\theintvl\curtextsize}% Заголовки отделяют от текста сверху и снизу тремя интервалами (ГОСТ Р 7.0.11-2011, 5.3.5).
\setaftersecskip{\theintvl\curtextsize}
\setbeforesubsecskip{\theintvl\curtextsize}
\setaftersubsecskip{\theintvl\curtextsize}
\setbeforesubsubsecskip{\theintvl\curtextsize}
\setaftersubsubsecskip{\theintvl\curtextsize}

%%% Блок дополнительного управления размерами заголовков
\ifnumequal{\value{headingsize}}{1}{% Пропорциональные заголовки и базовый шрифт 14 пт
    \renewcommand{\basegostsectionfont}{\large\bfseries}
    \renewcommand*{\chapnamefont}{\Large\bfseries}
    \renewcommand*{\chapnumfont}{\Large\bfseries}
    \renewcommand*{\chaptitlefont}{\Large\bfseries}
}{}

%%% Счётчики %%%

%% Упрощённые настройки шаблона диссертации: нумерация формул, таблиц, рисунков
\ifnumequal{\value{contnumeq}}{1}{%
    \counterwithout{equation}{chapter} % Убираем связанность номера формулы с номером главы/раздела
}{}
\ifnumequal{\value{contnumfig}}{1}{%
    \counterwithout{figure}{chapter}   % Убираем связанность номера рисунка с номером главы/раздела
}{}
\ifnumequal{\value{contnumtab}}{1}{%
    \counterwithout{table}{chapter}    % Убираем связанность номера таблицы с номером главы/раздела
}{}


%%http://www.linux.org.ru/forum/general/6993203#comment-6994589 (используется totcount)
\makeatletter
\def\formbytotal#1#2#3#4#5{%
    \newcount\@c
    \@c\totvalue{#1}\relax
    \newcount\@last
    \newcount\@pnul
    \@last\@c\relax
    \divide\@last 10
    \@pnul\@last\relax
    \divide\@pnul 10
    \multiply\@pnul-10
    \advance\@pnul\@last
    \multiply\@last-10
    \advance\@last\@c
    \total{#1}~#2%
    \ifnum\@pnul=1#5\else%
    \ifcase\@last#5\or#3\or#4\or#4\or#4\else#5\fi
    \fi
}
\makeatother

\AtBeginDocument{
%% регистрируем счётчики в системе totcounter
    \regtotcounter{totalcount@figure}
    \regtotcounter{totalcount@table}       % Если иным способом поставить в преамбуле то ошибка в числе таблиц
    \regtotcounter{TotPages}               % Если иным способом поставить в преамбуле то ошибка в числе страниц
}

%%% Правильная нумерация приложений %%%
%% По ГОСТ 2.105, п. 4.3.8 Приложения обозначают заглавными буквами русского алфавита,
%% начиная с А, за исключением букв Ё, З, Й, О, Ч, Ь, Ы, Ъ.
%% Здесь также переделаны все нумерации русскими буквами.
\ifxetexorluatex
    \makeatletter
    \def\russian@Alph#1{\ifcase#1\or
       А\or Б\or В\or Г\or Д\or Е\or Ж\or
       И\or К\or Л\or М\or Н\or
       П\or Р\or С\or Т\or У\or Ф\or Х\or
       Ц\or Ш\or Щ\or Э\or Ю\or Я\else\xpg@ill@value{#1}{russian@Alph}\fi}
    \def\russian@alph#1{\ifcase#1\or
       а\or б\or в\or г\or д\or е\or ж\or
       и\or к\or л\or м\or н\or
       п\or р\or с\or т\or у\or ф\or х\or
       ц\or ш\or щ\or э\or ю\or я\else\xpg@ill@value{#1}{russian@alph}\fi}
    \makeatother
\else
    \makeatletter
    \if@uni@ode
      \def\russian@Alph#1{\ifcase#1\or
        А\or Б\or В\or Г\or Д\or Е\or Ж\or
        И\or К\or Л\or М\or Н\or
        П\or Р\or С\or Т\or У\or Ф\or Х\or
        Ц\or Ш\or Щ\or Э\or Ю\or Я\else\@ctrerr\fi}
    \else
      \def\russian@Alph#1{\ifcase#1\or
        \CYRA\or\CYRB\or\CYRV\or\CYRG\or\CYRD\or\CYRE\or\CYRZH\or
        \CYRI\or\CYRK\or\CYRL\or\CYRM\or\CYRN\or
        \CYRP\or\CYRR\or\CYRS\or\CYRT\or\CYRU\or\CYRF\or\CYRH\or
        \CYRC\or\CYRSH\or\CYRSHCH\or\CYREREV\or\CYRYU\or
        \CYRYA\else\@ctrerr\fi}
    \fi
    \if@uni@ode
      \def\russian@alph#1{\ifcase#1\or
        а\or б\or в\or г\or д\or е\or ж\or
        и\or к\or л\or м\or н\or
        п\or р\or с\or т\or у\or ф\or х\or
        ц\or ш\or щ\or э\or ю\or я\else\@ctrerr\fi}
    \else
      \def\russian@alph#1{\ifcase#1\or
        \cyra\or\cyrb\or\cyrv\or\cyrg\or\cyrd\or\cyre\or\cyrzh\or
        \cyri\or\cyrk\or\cyrl\or\cyrm\or\cyrn\or
        \cyrp\or\cyrr\or\cyrs\or\cyrt\or\cyru\or\cyrf\or\cyrh\or
        \cyrc\or\cyrsh\or\cyrshch\or\cyrerev\or\cyryu\or
        \cyrya\else\@ctrerr\fi}
    \fi
    \makeatother
\fi
  % Стили для диссертации
\input{Dissertation/userstyles} % Стили для специфических пользовательских задач

%%% Библиография. Выбор движка для реализации %%%
\ifnumequal{\value{bibliosel}}{1}{%
    \input{biblio/predefined}   % Встроенная реализация с загрузкой файла через движок bibtex8
}{
    \input{biblio/biblatex}     % Реализация пакетом biblatex через движок biber
}

% Вывести информацию о выбранных опциях в лог сборки
%\typeout{Selected options:}
%\typeout{Draft mode: \arabic{draft}}
%\typeout{Font: \arabic{fontfamily}}
%\typeout{AltFont: \arabic{usealtfont}}
%\typeout{Bibliography backend: \arabic{bibliosel}}
%\typeout{Precompile images: \arabic{imgprecompile}}
% Вывести информацию о версиях используемых библиотек в лог сборки
\listfiles

%%% Управление компиляцией отдельных частей диссертации %%%
% Необходимо сначала иметь полностью скомпилированный документ, чтобы все
% промежуточные файлы были в наличии
% Затем, для вывода отдельных частей можно воспользоваться командой \includeonly
% Ниже примеры использования команды:
%
%\includeonly{Dissertation/part2}
%\includeonly{Dissertation/contents,Dissertation/appendix,Dissertation/conclusion}
%
% Если все команды закомментированы, то документ будет выведен в PDF файл полностью

\begin{document}

%%% Переопределение именований %%%
\renewcommand{\contentsname}{Содержание} % (ГОСТ Р 7.0.11-2011, 4)
\renewcommand{\figurename}{Рисунок} % (ГОСТ Р 7.0.11-2011, 5.3.9)
\renewcommand{\tablename}{Таблица} % (ГОСТ Р 7.0.11-2011, 5.3.10)
\renewcommand{\listfigurename}{Список рисунков}
\renewcommand{\listtablename}{Список таблиц}
\renewcommand{\bibname}{\bibtitlefull}
                 % Переопределение именований

%%% Структура диссертации (ГОСТ Р 7.0.11-2011, 4)
\setcounter{page}{5}
% Оглавление (ГОСТ Р 7.0.11-2011, 5.2)
\ifdefmacro{\microtypesetup}{\microtypesetup{protrusion=false}}{} % не рекомендуется применять пакет микротипографики к автоматически генерируемому оглавлению
\tableofcontents*
\addtocontents{toc}{\protect}
%\endTOCtrue
\ifdefmacro{\microtypesetup}{\microtypesetup{protrusion=true}}{}
\addcontentsline{toc}{chapter}{Содержание}        % Оглавление
\chapter*{Реферат}
\addcontentsline{toc}{chapter}{Реферат} 
\section*{Актуальность темы}
\noindent Текст, первый абзац без отступа. Реферат схож с аннотацией к научной статье, но должна быть более детальной

Следующий абзац с отступом. Кратко повторите основное содержание диссертации, включая мотивацию для этой работы, новизну по сравнению с предыдущими работами, описание проблемы, методологию, результаты и выводы.
\section*{Научная новизна}
\section*{Научные положения, выносимые на защиту:}
\section*{Практическая значимость}
\section*{Достоверность}
\section*{Внедрение результатов работы}
\section*{Публикации}
\section*{Личный вклад автора}
\section*{Структура и объем диссертации}
\section*{ОСНОВНОЕ СОДЕРЖАНИЕ РАБОТЫ}
\chapter*{Synopsis}
\addcontentsline{toc}{chapter}{Synopsis} 
\section*{Relevance}
\noindent Текст, первый абзац без отступа. Реферат схож с аннотацией к научной статье, но должна быть более детальной

Следующий абзац с отступом. Кратко повторите основное содержание диссертации, включая мотивацию для этой работы, новизну по сравнению с предыдущими работами, описание проблемы, методологию, результаты и выводы.
\section*{The goal}
\section*{Scientific tasks:}
\section*{Scientific novelty}
\section*{Scientific statements:}
\section*{Practical importance}
\section*{Reliability and the validity}
\section*{Implementation of the obtained results}
\section*{Approbation}
\section*{Publication}
\section*{Author contribution}
\section*{The structure}
\section*{MAIN CONTENTS OF WORK}

\include{Dissertation/acronyms}        % Список сокращений
\chapter*{Термины (необязательно)} % Заголовок
\addcontentsline{toc}{chapter}{Термины (необязательно)}  % Добавляем его в оглавление
\noindent
%\begin{longtabu} to \dimexpr \textwidth-5\tabcolsep {r X}
\begin{longtabu} to \textwidth {r X}
% Жирное начертание для математических символов может иметь
% дополнительный смысл, поэтому они приводятся как в тексте
% диссертации
\(\begin{rcases}
a_n\\
b_n
\end{rcases}\)  &
\begin{minipage}{\linewidth}
коэффициенты разложения Ми в дальнем поле соответствующие
электрическим и магнитным мультиполям
\end{minipage}
\\
\({\boldsymbol{\hat{\mathrm e}}}\) & единичный вектор \\
\(E_0\) & амплитуда падающего поля\\

\end{longtabu}
\addtocounter{table}{-1}% Нужно откатить на единицу счетчик номеров таблиц, так как предыдующая таблица сделана для удобства представления информации по ГОСТ
         % Список терминов
\chapter*{Условные обозначения (необязательно)} % Заголовок
\addcontentsline{toc}{chapter}{Условные обозначения (необязательно)}  % Добавляем его в оглавление
\noindent
%\begin{longtabu} to \dimexpr \textwidth-5\tabcolsep {r X}
\begin{longtabu} to \textwidth {r X}
% Жирное начертание для математических символов может иметь
% дополнительный смысл, поэтому они приводятся как в тексте
% диссертации
\(\begin{rcases}
a_n\\
b_n
\end{rcases}\)  &
\begin{minipage}{\linewidth}
коэффициенты разложения Ми в дальнем поле соответствующие
электрическим и магнитным мультиполям
\end{minipage}
\\
\({\boldsymbol{\hat{\mathrm e}}}\) & единичный вектор \\
\(E_0\) & амплитуда падающего поля\\

\end{longtabu}
\addtocounter{table}{-1}% Нужно откатить на единицу счетчик номеров таблиц, так как предыдующая таблица сделана для удобства представления информации по ГОСТ
       % Список условных обозначений
\chapter*{Введение}                         % Заголовок
\addcontentsline{toc}{chapter}{Введение}
\noindent Текст, первый абзац без отступа. Во введении в диссертацию нужно отразить актуальность темы исследования и причины выбора темы, цели и задачи диссертации, проблему, подлежащую исследованию, методы, использованные в диссертации, использованную исходную информацию, теоретическую и практическую новизну исследования, обзор апробации результатов исследования (включая конференции и семинары, опубликованные статьи и т.д.).
Следующий параграф с отступом.
    % Введение
\chapter{Оформление различных элементов}\label{ch:ch1}

Цитируем \cite{confbib1}

Вот так пишется нумерованная формула:
\begin{equation}
  \label{eq:equation1}
  e = \lim_{n \to \infty} \left( 1+\frac{1}{n} \right) ^n
\end{equation}

Нумерованных формул может быть несколько:
\begin{equation}
  \label{eq:equation2}
  \lim_{n \to \infty} \sum_{k=1}^n \frac{1}{k^2} = \frac{\pi^2}{6}
\end{equation}

Впоследствии на формулы~\eqref{eq:equation1} и~\eqref{eq:equation2} можно ссылаться.

Сделать так, чтобы номер формулы стоял напротив средней строки, можно,
используя окружение \verb|multlined| (пакет \verb|mathtools|) вместо
\verb|multline| внутри окружения \verb|equation|. Вот так:
\begin{equation} % \tag{S} % tag - вписывает свой текст
  \label{eq:equation3}
    \begin{multlined}
        1+ 2+3+4+5+6+7+\dots + \\
        + 50+51+52+53+54+55+56+57 + \dots + \\
        + 96+97+98+99+100=5050
    \end{multlined}
\end{equation}

Используя команду \verb|\labelcref| из пакета \verb|cleveref|, можно
красиво ссылаться сразу на несколько формул
(\labelcref{eq:equation1, eq:equation3, eq:equation2}), даже перепутав
порядок ссылок \verb|(\labelcref{eq:equation1, eq:equation3, eq:equation2})|.
           % Глава 1
\chapter{Длинное название главы, в которой мы смотрим на~примеры того, как будут верстаться изображения и~списки}\label{ch:ch2}

\section{Одиночное изображение}\label{sec:ch2/sec1}

\begin{figure}[ht]
  \centerfloat{
    \includegraphics[scale=0.27]{latex}
  }
  \caption{TeX.}\label{fig:latex}
\end{figure}

Для выравнивания изображения по-центру используется команда \verb+\centerfloat+, которая является во
многом улучшенной версией встроенной команды \verb+\centering+.

\section{Длинное название параграфа, в котором мы узнаём как сделать две картинки с~общим номером и названием}\label{sec:ch2/sect2}

А это две картинки под общим номером и названием:
\begin{figure}[ht]
  \begin{minipage}[b][][b]{0.49\linewidth}\centering
    \includegraphics[width=0.5\linewidth]{knuth1} \\ а)
  \end{minipage}
  \hfill
  \begin{minipage}[b][][b]{0.49\linewidth}\centering
    \includegraphics[width=0.5\linewidth]{knuth2} \\ б)
  \end{minipage}
  \caption{Очень длинная подпись к изображению,
      на котором представлены две фотографии Дональда Кнута}
  \label{fig:knuth}
\end{figure}

Те~же~две картинки под~общим номером и~названием,
но с автоматизированной нумерацией подрисунков:
\begin{figure}[ht]
    \centerfloat{
        \hfill
        \subbottom[List-of-Figures entry][Первый подрисунок\label{fig:knuth_2-1}]{%
            \includegraphics[width=0.25\linewidth]{knuth1}}
        \hfill
        \subbottom[\label{fig:knuth_2-2}]{%
            \includegraphics[width=0.25\linewidth]{knuth2}}
        \hfill
        \subbottom[Третий подрисунок]{%
            \includegraphics[width=0.3\linewidth]{example-image-c}}
        \hfill
    }
    \legend{Подрисуночный текст, описывающий обозначения, например. Согласно
    ГОСТ 2.105, пункт 4.3.1, располагается перед наименованием рисунка.}
    \caption[Этот текст попадает в названия рисунков в списке рисунков]{Очень
    длинная подпись к второму изображению, на~котором представлены две
    фотографии Дональда Кнута}\label{fig:knuth_2}
\end{figure}

На рисунке~\ref{fig:knuth_2-1} показан Дональд Кнут без головного убора.
На рисунке~\ref{fig:knuth_2}\subcaptionref*{fig:knuth_2-2}
показан Дональд Кнут в головном уборе.
           % Глава 2
\chapter{Вёрстка таблиц}\label{ch:ch3}

\section{Таблица обыкновенная}\label{sec:ch3/sect1}

Так размещается таблица:

\begin{table} [htbp]
  \centering
  \changecaptionwidth\captionwidth{15cm}
  \caption{Название таблицы}\label{tab:Ts0Sib}%
  \begin{tabular}{| p{3cm} || p{3cm} | p{3cm} | p{4cm}l |}
  \hline
  \hline
  Месяц   & \centering \(T_{min}\), К & \centering \(T_{max}\), К &\centering  \((T_{max} - T_{min})\), К & \\
  \hline
  Декабрь &\centering  253.575   &\centering  257.778    &\centering      4.203  &   \\
  Январь  &\centering  262.431   &\centering  263.214    &\centering      0.783  &   \\
  Февраль &\centering  261.184   &\centering  260.381    &\centering     \(-\)0.803  &   \\
  \hline
  \hline
  \end{tabular}
\end{table}

\begin{table} [htbp]% Пример записи таблицы с номером, но без отображаемого наименования
    \centering
    \parbox{9cm}{% чтобы лучше смотрелось, подбирается самостоятельно
        \captiondelim{}% должен стоять до самого пустого caption
        \caption{}%
        \label{tab:test1}%
        \begin{SingleSpace}
            \begin{tabular}{| c | c | c | c |}
                \hline
                Оконная функция & \({2N}\)& \({4N}\)& \({8N}\)\\ \hline
                Прямоугольное   & 8.72  & 8.77  & 8.77  \\ \hline
                Ханна           & 7.96  & 7.93  & 7.93  \\ \hline
                Хэмминга        & 8.72  & 8.77  & 8.77  \\ \hline
                Блэкмана        & 8.72  & 8.77  & 8.77  \\ \hline
            \end{tabular}%
        \end{SingleSpace}
    }
\end{table}

Таблица~\ref{tab:test2} "--- пример таблицы, оформленной в~классическом книжном
варианте или~очень близко к~нему. \mbox{ГОСТу} по~сути не~противоречит. Можно
ещё~улучшить представление, с~помощью пакета \verb|siunitx| или~подобного.

\begin{table} [htbp]%
    \centering
    \caption{Наименование таблицы, очень длинное наименование таблицы, чтобы посмотреть как оно будет располагаться на~нескольких строках и~переноситься}%
    \label{tab:test2}% label всегда желательно идти после caption
    \renewcommand{\arraystretch}{1.5}%% Увеличение расстояния между рядами, для улучшения восприятия.
    \begin{SingleSpace}
        \begin{tabular}{@{}@{\extracolsep{20pt}}llll@{}} %Вертикальные полосы не используются принципиально, как и лишние горизонтальные (допускается по ГОСТ 2.105 пункт 4.4.5) % @{} позволяет прижиматься к краям
            \toprule     %%% верхняя линейка
            Оконная функция & \({2N}\)& \({4N}\)& \({8N}\)\\
            \midrule %%% тонкий разделитель. Отделяет названия столбцов. Обязателен по ГОСТ 2.105 пункт 4.4.5
            Прямоугольное   & 8.72  & 8.77  & 8.77  \\
            Ханна           & 7.96  & 7.93  & 7.93  \\
            Хэмминга        & 8.72  & 8.77  & 8.77  \\
            Блэкмана        & 8.72  & 8.77  & 8.77  \\
            \bottomrule %%% нижняя линейка
        \end{tabular}%
    \end{SingleSpace}
\end{table}

\section{Таблица с многострочными ячейками и примечанием}

В таблице~\ref{tab:makecell} приведён пример использования команды
\verb+\multicolumn+ для объединения горизонтальных ячеек таблицы,
и команд пакета \textit{makecell} для добавления разрыва строки внутри ячеек.

\begin{table} [htbp]
	\centering
	\caption{Пример использования функций пакета \textit{makecell}.}%
	\label{tab:makecell}%
	\begin{tabular}{| c | c | c | c |}
	  \hline
	  Колонка 1                                    & Колонка 2        & \thead{Название колонки 3, \\ не помещающееся в одну строку} & Колонка 4 \\ \hline
	  \multicolumn{4}{|c|}{Выравнивание по центру}                                                                                               \\ \hline
	  \multicolumn{2}{|r|}{\makecell{Выравнивание к \\ правому краю}} & \multicolumn{2}{|l|}{Выравнивание к левому краю}                         \\ \hline
	  \makecell{В этой ячейке \\ много информации} & 8.72             & 8.55                                                         & 8.44      \\ \cline{3-4}
	  А в этой мало                                & 8.22             & \multicolumn{2}{|c|}{5}                                                  \\ \hline
	\end{tabular}%
\end{table}


\section{Параграф "--- два}\label{sec:ch3/sect2}

Некоторый текст.

\section{Параграф с подпараграфами}\label{sec:ch3/sect3}

\subsection{Подпараграф "--- один}\label{subsec:ch3/sect3/sub1}

Некоторый текст.

\subsection{Подпараграф "--- два}\label{subsec:ch3/sect3/sub2}

Некоторый текст.

\clearpage           % Глава 3
\include{Dissertation/lists}           % Список таблиц и изображений
\include{Dissertation/references}
\chapter*{Благодарности}
\addcontentsline{toc}{chapter}{Благодарности}
\noindent Здесь Вы можете поблагодарить своего научного руководителя, коллег, членов семьи и т.д. за поддержку.
Не забудьте также упомянуть все финансирующие организации, которые оказывали поддержку Вашему исследованию.

	

\chapter*{Приложение 1}
\addcontentsline{toc}{chapter}{Приложение 1}


	
\chapter*{Приложение 2}
\addcontentsline{toc}{chapter}{Приложение 2}

	
\chapter*{Публикации автора по теме диссертации}
\addcontentsline{toc}{chapter}{Публикации автора по теме диссертации}
Список публикаций автора, на основании которых подготовлена диссертация:

\begin{enumerate}[label=\Roman*., leftmargin=\parindent, align=left]
	\item Статья 1
	\item Статья 2
	\item Статья 3
	\item
\end{enumerate}
	    % Список публикаций
%%% Настройки для приложений
\appendix
% Оформление заголовков приложений ближе к ГОСТ:
\setlength{\midchapskip}{20pt}
\renewcommand*{\afterchapternum}{\par\nobreak\vskip \midchapskip}

%\chapter{G}\label{app:A}

Для крупных листингов есть два способа. Первый красивый, но в нём могут быть
проблемы с поддержкой кириллицы (у вас может встречаться в~комментариях
и печатаемых сообщениях), он представлен на листинге~\ref{lst:hwbeauty}.
\begin{ListingEnv}[!h]% настройки floating аналогичны окружению figure
    \captiondelim{ } % разделитель идентификатора с номером от наименования
    \caption{Программа ,,Hello, world`` на \protect\cpp}\label{lst:hwbeauty}
    % окружение учитывает пробелы и табуляции и применяет их в сответсвии с настройками
    \begin{lstlisting}[language={[ISO]C++}]
	#include <iostream>
	using namespace std;

	int main() //кириллица в комментариях при xelatex и lualatex имеет проблемы с пробелами
	{
		cout << "Hello, world" << endl; //latin letters in commentaries
		system("pause");
		return 0;
	}
    \end{lstlisting}
\end{ListingEnv}%
Второй не~такой красивый, но без ограничений (см.~листинг~\ref{lst:hwplain}).
\begin{ListingEnv}[!h]
    \captiondelim{ } % разделитель идентификатора с номером от наименования
    \caption{Программа ,,Hello, world`` без подсветки}\label{lst:hwplain}
    \begin{Verb}

        #include <iostream>
        using namespace std;

        int main() //кириллица в комментариях
        {
            cout << "Привет, мир" << endl;
        }
    \end{Verb}
\end{ListingEnv}

Можно использовать первый для вставки небольших фрагментов
внутри текста, а второй для вставки полного
кода в приложении, если таковое имеется.

Если нужно вставить совсем короткий пример кода (одна или две строки),
то~выделение  линейками и нумерация может смотреться чересчур громоздко.
В таких случаях можно использовать окружения \texttt{lstlisting} или
\texttt{Verb} без \texttt{ListingEnv}. Приведём такой пример
с указанием языка программирования, отличного от~заданного по умолчанию:
\begin{lstlisting}[language=Haskell]
fibs = 0 : 1 : zipWith (+) fibs (tail fibs)
\end{lstlisting}
Такое решение~--- со вставкой нумерованных листингов покрупнее
и вставок без выделения для маленьких фрагментов~--- выбрано,
например, в книге Эндрю Таненбаума и Тодда Остина по архитектуре
%компьютера~\autocite{TanAus2013} (см.~рис.~\ref{fig:tan-aus}).

Наконец, для оформления идентификаторов внутри строк
(функция \lstinline{main} и~тому подобное) используется
\texttt{lstinline} или, самое простое, моноширинный текст
(\texttt{\textbackslash texttt}).

Пример~\ref{lst:internal3}, иллюстрирующий подключение переопределённого
языка. Может быть полезным, если подсветка кода работает криво. Без
дополнительного окружения, с подписью и ссылкой, реализованной встроенным
средством.
\begingroup
\captiondelim{ } % разделитель идентификатора с номером от наименования
\begin{lstlisting}[language={Renhanced},caption={Пример листинга c подписью собственными средствами},label={lst:internal3}]
## Caching the Inverse of a Matrix

## Matrix inversion is usually a costly computation and there may be some
## benefit to caching the inverse of a matrix rather than compute it repeatedly
## This is a pair of functions that cache the inverse of a matrix.

## makeCacheMatrix creates a special "matrix" object that can cache its inverse

makeCacheMatrix <- function(x = matrix()) {#кириллица в комментариях при xelatex и lualatex имеет проблемы с пробелами
    i <- NULL
    set <- function(y) {
        x <<- y
        i <<- NULL
    }
    get <- function() x
    setSolved <- function(solve) i <<- solve
    getSolved <- function() i
    list(set = set, get = get,
    setSolved = setSolved,
    getSolved = getSolved)

}


## cacheSolve computes the inverse of the special "matrix" returned by
## makeCacheMatrix above. If the inverse has already been calculated (and the
## matrix has not changed), then the cachesolve should retrieve the inverse from
## the cache.

cacheSolve <- function(x, ...) {
    ## Return a matrix that is the inverse of 'x'
    i <- x$getSolved()
    if(!is.null(i)) {
        message("getting cached data")
        return(i)
    }
    data <- x$get()
    i <- solve(data, ...)
    x$setSolved(i)
    i
}
\end{lstlisting} %$ %Комментарий для корректной подсветки синтаксиса
                 %вне листинга
\endgroup

Листинг~\ref{lst:external1} подгружается из внешнего файла. Приходится
загружать без окружения дополнительного. Иначе по страницам не переносится.
\begingroup
\captiondelim{ } % разделитель идентификатора с номером от наименования
    \lstinputlisting[lastline=78,language={R},caption={Листинг из внешнего файла},label={lst:external1}]{listings/run_analysis.R}
\endgroup

\chapter{Очень длинное название второго приложения, в~котором продемонстрирована работа с~длинными таблицами}\label{app:B}

\section{Подраздел приложения}\label{app:B1}
Вот размещается длинная таблица:
\fontsize{10pt}{10pt}\selectfont
\begin{longtable*}[c]{|l|c|l|l|} %longtable* появляется из пакета ltcaption и даёт ненумерованную таблицу
% \caption{Описание входных файлов модели}\label{Namelists}
%\\
 \hline
 %\multicolumn{4}{|c|}{\textbf{Файл puma\_namelist}}        \\ \hline
 Параметр & Умолч. & Тип & Описание               \\ \hline
                                              \endfirsthead   \hline
 \multicolumn{4}{|c|}{\small\slshape (продолжение)}        \\ \hline
 Параметр & Умолч. & Тип & Описание               \\ \hline
                                              \endhead        \hline
% \multicolumn{4}{|c|}{\small\slshape (окончание)}        \\ \hline
% Параметр & Умолч. & Тип & Описание               \\ \hline
%                                             \endlasthead        \hline
 \multicolumn{4}{|r|}{\small\slshape продолжение следует}  \\ \hline
                                              \endfoot        \hline
                                              \endlastfoot
 \multicolumn{4}{|l|}{\&INP}        \\ \hline
 kick & 1 & int & 0: инициализация без шума (\(p_s = const\)) \\
      &   &     & 1: генерация белого шума                  \\
      &   &     & 2: генерация белого шума симметрично относительно \\
  & & & экватора    \\
 mars & 0 & int & 1: инициализация модели для планеты Марс     \\
 kick & 1 & int & 0: инициализация без шума (\(p_s = const\)) \\
      &   &     & 1: генерация белого шума                  \\
      &   &     & 2: генерация белого шума симметрично относительно \\
  & & & экватора    \\
 mars & 0 & int & 1: инициализация модели для планеты Марс     \\
kick & 1 & int & 0: инициализация без шума (\(p_s = const\)) \\
      &   &     & 1: генерация белого шума                  \\
      &   &     & 2: генерация белого шума симметрично относительно \\
  & & & экватора    \\
 mars & 0 & int & 1: инициализация модели для планеты Марс     \\
kick & 1 & int & 0: инициализация без шума (\(p_s = const\)) \\
      &   &     & 1: генерация белого шума                  \\
      &   &     & 2: генерация белого шума симметрично относительно \\
  & & & экватора    \\
 mars & 0 & int & 1: инициализация модели для планеты Марс     \\
kick & 1 & int & 0: инициализация без шума (\(p_s = const\)) \\
      &   &     & 1: генерация белого шума                  \\
      &   &     & 2: генерация белого шума симметрично относительно \\
  & & & экватора    \\
 mars & 0 & int & 1: инициализация модели для планеты Марс     \\
kick & 1 & int & 0: инициализация без шума (\(p_s = const\)) \\
      &   &     & 1: генерация белого шума                  \\
      &   &     & 2: генерация белого шума симметрично относительно \\
  & & & экватора    \\
 mars & 0 & int & 1: инициализация модели для планеты Марс     \\
kick & 1 & int & 0: инициализация без шума (\(p_s = const\)) \\
      &   &     & 1: генерация белого шума                  \\
      &   &     & 2: генерация белого шума симметрично относительно \\
  & & & экватора    \\
 mars & 0 & int & 1: инициализация модели для планеты Марс     \\
kick & 1 & int & 0: инициализация без шума (\(p_s = const\)) \\
      &   &     & 1: генерация белого шума                  \\
      &   &     & 2: генерация белого шума симметрично относительно \\
  & & & экватора    \\
 mars & 0 & int & 1: инициализация модели для планеты Марс     \\
kick & 1 & int & 0: инициализация без шума (\(p_s = const\)) \\
      &   &     & 1: генерация белого шума                  \\
      &   &     & 2: генерация белого шума симметрично относительно \\
  & & & экватора    \\
 mars & 0 & int & 1: инициализация модели для планеты Марс     \\
kick & 1 & int & 0: инициализация без шума (\(p_s = const\)) \\
      &   &     & 1: генерация белого шума                  \\
      &   &     & 2: генерация белого шума симметрично относительно \\
  & & & экватора    \\
 mars & 0 & int & 1: инициализация модели для планеты Марс     \\
kick & 1 & int & 0: инициализация без шума (\(p_s = const\)) \\
      &   &     & 1: генерация белого шума                  \\
      &   &     & 2: генерация белого шума симметрично относительно \\
  & & & экватора    \\
 mars & 0 & int & 1: инициализация модели для планеты Марс     \\
kick & 1 & int & 0: инициализация без шума (\(p_s = const\)) \\
      &   &     & 1: генерация белого шума                  \\
      &   &     & 2: генерация белого шума симметрично относительно \\
  & & & экватора    \\
 mars & 0 & int & 1: инициализация модели для планеты Марс     \\
kick & 1 & int & 0: инициализация без шума (\(p_s = const\)) \\
      &   &     & 1: генерация белого шума                  \\
      &   &     & 2: генерация белого шума симметрично относительно \\
  & & & экватора    \\
 mars & 0 & int & 1: инициализация модели для планеты Марс     \\
kick & 1 & int & 0: инициализация без шума (\(p_s = const\)) \\
      &   &     & 1: генерация белого шума                  \\
      &   &     & 2: генерация белого шума симметрично относительно \\
  & & & экватора    \\
 mars & 0 & int & 1: инициализация модели для планеты Марс     \\
kick & 1 & int & 0: инициализация без шума (\(p_s = const\)) \\
      &   &     & 1: генерация белого шума                  \\
      &   &     & 2: генерация белого шума симметрично относительно \\
  & & & экватора    \\
 mars & 0 & int & 1: инициализация модели для планеты Марс     \\
 \hline
  %& & & \(\:\) \\
 \multicolumn{4}{|l|}{\&SURFPAR}        \\ \hline
kick & 1 & int & 0: инициализация без шума (\(p_s = const\)) \\
      &   &     & 1: генерация белого шума                  \\
      &   &     & 2: генерация белого шума симметрично относительно \\
  & & & экватора    \\
 mars & 0 & int & 1: инициализация модели для планеты Марс     \\
kick & 1 & int & 0: инициализация без шума (\(p_s = const\)) \\
      &   &     & 1: генерация белого шума                  \\
      &   &     & 2: генерация белого шума симметрично относительно \\
  & & & экватора    \\
 mars & 0 & int & 1: инициализация модели для планеты Марс     \\
kick & 1 & int & 0: инициализация без шума (\(p_s = const\)) \\
      &   &     & 1: генерация белого шума                  \\
      &   &     & 2: генерация белого шума симметрично относительно \\
  & & & экватора    \\
 mars & 0 & int & 1: инициализация модели для планеты Марс     \\
kick & 1 & int & 0: инициализация без шума (\(p_s = const\)) \\
      &   &     & 1: генерация белого шума                  \\
      &   &     & 2: генерация белого шума симметрично относительно \\
  & & & экватора    \\
 mars & 0 & int & 1: инициализация модели для планеты Марс     \\
kick & 1 & int & 0: инициализация без шума (\(p_s = const\)) \\
      &   &     & 1: генерация белого шума                  \\
      &   &     & 2: генерация белого шума симметрично относительно \\
  & & & экватора    \\
 mars & 0 & int & 1: инициализация модели для планеты Марс     \\
kick & 1 & int & 0: инициализация без шума (\(p_s = const\)) \\
      &   &     & 1: генерация белого шума                  \\
      &   &     & 2: генерация белого шума симметрично относительно \\
  & & & экватора    \\
 mars & 0 & int & 1: инициализация модели для планеты Марс     \\
kick & 1 & int & 0: инициализация без шума (\(p_s = const\)) \\
      &   &     & 1: генерация белого шума                  \\
      &   &     & 2: генерация белого шума симметрично относительно \\
  & & & экватора    \\
 mars & 0 & int & 1: инициализация модели для планеты Марс     \\
kick & 1 & int & 0: инициализация без шума (\(p_s = const\)) \\
      &   &     & 1: генерация белого шума                  \\
      &   &     & 2: генерация белого шума симметрично относительно \\
  & & & экватора    \\
 mars & 0 & int & 1: инициализация модели для планеты Марс     \\
kick & 1 & int & 0: инициализация без шума (\(p_s = const\)) \\
      &   &     & 1: генерация белого шума                  \\
      &   &     & 2: генерация белого шума симметрично относительно \\
  & & & экватора    \\
 mars & 0 & int & 1: инициализация модели для планеты Марс     \\
 \hline
\end{longtable*}

\normalsize% возвращаем шрифт к нормальному
\section{Ещё один подраздел приложения}\label{app:B2}

Нужно больше подразделов приложения!
Конвынёры витюпырата но нам, тебиквюэ мэнтётюм позтюлант ед про. Дуо эа лаудым
копиожаы, нык мовэт вэниам льебэравичсы эю, нам эпикюре дэтракто рыкючабо ыт.

Пример длинной таблицы с записью продолжения по ГОСТ 2.105:

\begingroup
    \centering
    \small
    \begin{longtable}[c]{|l|c|l|l|}
    \caption{Наименование таблицы средней длины}\label{tab:test5}% label всегда желательно идти после caption
    \\[-0.45\onelineskip]
    \hline
    Параметр & Умолч. & Тип & Описание\\ \hline
    \endfirsthead%
    \caption*{\tabcapalign Продолжение таблицы~\thetable}\\[-0.45\onelineskip]
    \hline
    Параметр & Умолч. & Тип & Описание\\ \hline
    \endhead
    \hline
    \endfoot
    \hline
     \endlastfoot
     \multicolumn{4}{|l|}{\&INP}        \\ \hline
     kick & 1 & int & 0: инициализация без шума (\(p_s = const\)) \\
          &   &     & 1: генерация белого шума                  \\
          &   &     & 2: генерация белого шума симметрично относительно \\
      & & & экватора    \\
     mars & 0 & int & 1: инициализация модели для планеты Марс     \\
     kick & 1 & int & 0: инициализация без шума (\(p_s = const\)) \\
          &   &     & 1: генерация белого шума                  \\
          &   &     & 2: генерация белого шума симметрично относительно \\
      & & & экватора    \\
     mars & 0 & int & 1: инициализация модели для планеты Марс     \\
    kick & 1 & int & 0: инициализация без шума (\(p_s = const\)) \\
          &   &     & 1: генерация белого шума                  \\
          &   &     & 2: генерация белого шума симметрично относительно \\
      & & & экватора    \\
     mars & 0 & int & 1: инициализация модели для планеты Марс     \\
    kick & 1 & int & 0: инициализация без шума (\(p_s = const\)) \\
          &   &     & 1: генерация белого шума                  \\
          &   &     & 2: генерация белого шума симметрично относительно \\
      & & & экватора    \\
     mars & 0 & int & 1: инициализация модели для планеты Марс     \\
    kick & 1 & int & 0: инициализация без шума (\(p_s = const\)) \\
          &   &     & 1: генерация белого шума                  \\
          &   &     & 2: генерация белого шума симметрично относительно \\
      & & & экватора    \\
     mars & 0 & int & 1: инициализация модели для планеты Марс     \\
    kick & 1 & int & 0: инициализация без шума (\(p_s = const\)) \\
          &   &     & 1: генерация белого шума                  \\
          &   &     & 2: генерация белого шума симметрично относительно \\
      & & & экватора    \\
     mars & 0 & int & 1: инициализация модели для планеты Марс     \\
    kick & 1 & int & 0: инициализация без шума (\(p_s = const\)) \\
          &   &     & 1: генерация белого шума                  \\
          &   &     & 2: генерация белого шума симметрично относительно \\
      & & & экватора    \\
     mars & 0 & int & 1: инициализация модели для планеты Марс     \\
    kick & 1 & int & 0: инициализация без шума (\(p_s = const\)) \\
          &   &     & 1: генерация белого шума                  \\
          &   &     & 2: генерация белого шума симметрично относительно \\
      & & & экватора    \\
     mars & 0 & int & 1: инициализация модели для планеты Марс     \\
    kick & 1 & int & 0: инициализация без шума (\(p_s = const\)) \\
          &   &     & 1: генерация белого шума                  \\
          &   &     & 2: генерация белого шума симметрично относительно \\
      & & & экватора    \\
     mars & 0 & int & 1: инициализация модели для планеты Марс     \\
    kick & 1 & int & 0: инициализация без шума (\(p_s = const\)) \\
          &   &     & 1: генерация белого шума                  \\
          &   &     & 2: генерация белого шума симметрично относительно \\
      & & & экватора    \\
     mars & 0 & int & 1: инициализация модели для планеты Марс     \\
    kick & 1 & int & 0: инициализация без шума (\(p_s = const\)) \\
          &   &     & 1: генерация белого шума                  \\
          &   &     & 2: генерация белого шума симметрично относительно \\
      & & & экватора    \\
     mars & 0 & int & 1: инициализация модели для планеты Марс     \\
    kick & 1 & int & 0: инициализация без шума (\(p_s = const\)) \\
          &   &     & 1: генерация белого шума                  \\
          &   &     & 2: генерация белого шума симметрично относительно \\
      & & & экватора    \\
     mars & 0 & int & 1: инициализация модели для планеты Марс     \\
    kick & 1 & int & 0: инициализация без шума (\(p_s = const\)) \\
          &   &     & 1: генерация белого шума                  \\
          &   &     & 2: генерация белого шума симметрично относительно \\
      & & & экватора    \\
     mars & 0 & int & 1: инициализация модели для планеты Марс     \\
    kick & 1 & int & 0: инициализация без шума (\(p_s = const\)) \\
          &   &     & 1: генерация белого шума                  \\
          &   &     & 2: генерация белого шума симметрично относительно \\
      & & & экватора    \\
     mars & 0 & int & 1: инициализация модели для планеты Марс     \\
    kick & 1 & int & 0: инициализация без шума (\(p_s = const\)) \\
          &   &     & 1: генерация белого шума                  \\
          &   &     & 2: генерация белого шума симметрично относительно \\
      & & & экватора    \\
     mars & 0 & int & 1: инициализация модели для планеты Марс     \\
     \hline
      %& & & $\:$ \\
     \multicolumn{4}{|l|}{\&SURFPAR}        \\ \hline
    kick & 1 & int & 0: инициализация без шума (\(p_s = const\)) \\
          &   &     & 1: генерация белого шума                  \\
          &   &     & 2: генерация белого шума симметрично относительно \\
      & & & экватора    \\
     mars & 0 & int & 1: инициализация модели для планеты Марс     \\
    kick & 1 & int & 0: инициализация без шума (\(p_s = const\)) \\
          &   &     & 1: генерация белого шума                  \\
          &   &     & 2: генерация белого шума симметрично относительно \\
      & & & экватора    \\
     mars & 0 & int & 1: инициализация модели для планеты Марс     \\
    kick & 1 & int & 0: инициализация без шума (\(p_s = const\)) \\
          &   &     & 1: генерация белого шума                  \\
          &   &     & 2: генерация белого шума симметрично относительно \\
      & & & экватора    \\
     mars & 0 & int & 1: инициализация модели для планеты Марс     \\
    kick & 1 & int & 0: инициализация без шума (\(p_s = const\)) \\
          &   &     & 1: генерация белого шума                  \\
          &   &     & 2: генерация белого шума симметрично относительно \\
      & & & экватора    \\
     mars & 0 & int & 1: инициализация модели для планеты Марс     \\
    kick & 1 & int & 0: инициализация без шума (\(p_s = const\)) \\
          &   &     & 1: генерация белого шума                  \\
          &   &     & 2: генерация белого шума симметрично относительно \\
      & & & экватора    \\
     mars & 0 & int & 1: инициализация модели для планеты Марс     \\
    kick & 1 & int & 0: инициализация без шума (\(p_s = const\)) \\
          &   &     & 1: генерация белого шума                  \\
          &   &     & 2: генерация белого шума симметрично относительно \\
      & & & экватора    \\
     mars & 0 & int & 1: инициализация модели для планеты Марс     \\
    kick & 1 & int & 0: инициализация без шума (\(p_s = const\)) \\
          &   &     & 1: генерация белого шума                  \\
          &   &     & 2: генерация белого шума симметрично относительно \\
      & & & экватора    \\
     mars & 0 & int & 1: инициализация модели для планеты Марс     \\
    kick & 1 & int & 0: инициализация без шума (\(p_s = const\)) \\
          &   &     & 1: генерация белого шума                  \\
          &   &     & 2: генерация белого шума симметрично относительно \\
      & & & экватора    \\
     mars & 0 & int & 1: инициализация модели для планеты Марс     \\
    kick & 1 & int & 0: инициализация без шума (\(p_s = const\)) \\
          &   &     & 1: генерация белого шума                  \\
          &   &     & 2: генерация белого шума симметрично относительно \\
      & & & экватора    \\
     mars & 0 & int & 1: инициализация модели для планеты Марс     \\
    \end{longtable}
\normalsize% возвращаем шрифт к нормальному
\endgroup
\section{Использование длинных таблиц с окружением \textit{longtabu}}\label{app:B2a}

В таблице~\ref{tab:test-functions} более книжный вариант
длинной таблицы, используя окружение \verb!longtabu! и разнообразные
\verb!toprule! \verb!midrule! \verb!bottomrule! из~пакета
\verb!booktabs!. Чтобы визуально таблица смотрелась лучше, можно
использовать следующие параметры: в самом начале задаётся расстояние
между строчками с~помощью \verb!arraystretch!. Таблица задаётся на
всю ширину, \verb!longtabu! позволяет делить ширину колонок
пропорционально "--- тут три колонки в~пропорции 1.1:1:4 "--- для каждой
колонки первый параметр в~описании \verb!X[]!. Кроме того, в~таблице
убраны отступы слева и справа с~помощью \verb!@{}!
в~преамбуле таблицы. К~первому и~второму столбцу применяется
модификатор

\verb!>{\setlength{\baselineskip}{0.7\baselineskip}}!,

\noindent который уменьшает межстрочный интервал в для текста таблиц (иначе
заголовок второго столбца значительно шире, а двухстрочное имя
сливается с~окружающими). Для первой и второй колонки текст в ячейках
выравниваются по~центру как по~вертикали, так и по горизонтали "---
задаётся буквами \verb!m!~и~\verb!c!~в~описании столбца \verb!X[]!.

Так как формулы большие "--- используется окружение \verb!alignedat!,
чтобы отступ был одинаковый у всех формул "--- он сделан для всех, хотя
для большей части можно было и не использовать.  Чтобы формулы
занимали поменьше места в~каждом столбце формулы (где надо)
используется \verb!\textstyle! "--- он~делает дроби меньше, у~знаков
суммы и произведения "--- индексы сбоку. Иногда формулы слишком большая,
сливается со следующей, поэтому после неё ставится небольшой
дополнительный отступ \verb!\vspace*{2ex}!  Для штрафных функций "---
размер фигурных скобок задан вручную \verb!\Big\{!, т.\:к. не~умеет
\verb!alignedat! работать с~\verb!\left! и~\verb!\right! через
несколько строк/колонок.

В примечании к таблице наоборот, окружение \verb!cases! даёт слишком
большие промежутки между вариантами, чтобы их уменьшить, в конце
каждой строчки окружения использовался отрицательный дополнительный
отступ \verb!\\[-0.5em]!.

\begingroup % Ограничиваем область видимости arraystretch
\renewcommand{\arraystretch}{1.6}%% Увеличение расстояния между рядами, для улучшения восприятия.
\begin{longtabu} to \textwidth
{%
@{}>{\setlength{\baselineskip}{0.7\baselineskip}}X[1.1mc]%
>{\setlength{\baselineskip}{0.7\baselineskip}}X[1.1mc]%
X[4]@{}%
}
    \caption{Тестовые функции для оптимизации, \(D\) "---
      размерность. Для всех функций значение в точке глобального
      минимума равно нулю.\label{tab:test-functions}}\\% label всегда желательно идти после caption

    \toprule     %%% верхняя линейка
    Имя           &Стартовый диапазон параметров &Функция  \\
    \midrule %%% тонкий разделитель. Отделяет названия столбцов. Обязателен по ГОСТ 2.105 пункт 4.4.5
    \endfirsthead

    \multicolumn{3}{c}{\small\slshape (продолжение)}        \\
    \toprule     %%% верхняя линейка
    Имя           &Стартовый диапазон параметров &Функция  \\
    \midrule %%% тонкий разделитель. Отделяет названия столбцов. Обязателен по ГОСТ 2.105 пункт 4.4.5
    \endhead

    \multicolumn{3}{c}{\small\slshape (окончание)}        \\
    \toprule     %%% верхняя линейка
    Имя           &Стартовый диапазон параметров &Функция  \\
    \midrule %%% тонкий разделитель. Отделяет названия столбцов. Обязателен по ГОСТ 2.105 пункт 4.4.5
    \endlasthead

    \bottomrule %%% нижняя линейка
    \multicolumn{3}{r}{\small\slshape продолжение следует}  \\
    \endfoot
    \endlastfoot

    сфера         &\(\left[-100,\,100\right]^D\)   &
        \(\begin{aligned}
            \textstyle f_1(x)=\sum_{i=1}^Dx_i^2
        \end{aligned}\) \\
    Schwefel 2.22 &\(\left[-10,\,10\right]^D\)     &
        \(\begin{aligned}
            \textstyle f_2(x)=\sum_{i=1}^D|x_i|+\prod_{i=1}^D|x_i|
        \end{aligned}\) \\
    Schwefel 1.2  &\(\left[-100,\,100\right]^D\)   &
        \(\begin{aligned}
            \textstyle f_3(x)=\sum_{i=1}^D\left(\sum_{j=1}^ix_j\right)^2
        \end{aligned}\) \\
    Schwefel 2.21 &\(\left[-100,\,100\right]^D\)   &
        \(\begin{aligned}
            \textstyle f_4(x)=\max_i\!\left\{\left|x_i\right|\right\}
        \end{aligned}\) \\
    Rosenbrock    &\(\left[-30,\,30\right]^D\)     &
        \(\begin{aligned}
            \textstyle f_5(x)=
            \sum_{i=1}^{D-1}
            \left[100\!\left(x_{i+1}-x_i^2\right)^2+(x_i-1)^2\right]
        \end{aligned}\) \\
    ступенчатая   &\(\left[-100,\,100\right]^D\)   &
        \(\begin{aligned}
            \textstyle f_6(x)=\sum_{i=1}^D\big\lfloor x_i+0.5\big\rfloor^2
        \end{aligned}\) \\
    зашумлённая квартическая &\(\left[-1.28,\,1.28\right]^D\) &
        \(\begin{aligned}
            \textstyle f_7(x)=\sum_{i=1}^Dix_i^4+rand[0,1)
        \end{aligned}\)\vspace*{2ex}\\
    Schwefel 2.26 &\(\left[-500,\,500\right]^D\)   &
        \(\begin{aligned}
        f_8(x)= &\textstyle\sum_{i=1}^D-x_i\,\sin\sqrt{|x_i|}\,+ \\
                &\vphantom{\sum}+ D\cdot
                418.98288727243369
        \end{aligned}\)\\
    Rastrigin     &\(\left[-5.12,\,5.12\right]^D\) &
    \(\begin{aligned}
        \textstyle f_9(x)=\sum_{i=1}^D\left[x_i^2-10\,\cos(2\pi x_i)+10\right]
    \end{aligned}\)\vspace*{2ex}\\
    Ackley        &\(\left[-32,\,32\right]^D\)     &
        \(\begin{aligned}
            f_{10}(x)= &\textstyle -20\, \exp\!\left(
                            -0.2\sqrt{\frac{1}{D}\sum_{i=1}^Dx_i^2} \right)-\\
                       &\textstyle - \exp\left(
                            \frac{1}{D}\sum_{i=1}^D\cos(2\pi x_i)  \right)
                       + 20 + e
        \end{aligned}\) \\
    Griewank      &\(\left[-600,\,600\right]^D\) &
        \(\begin{aligned}
            f_{11}(x)= &\textstyle \frac{1}{4000}\sum_{i=1}^{D}x_i^2 -
                \prod_{i=1}^D\cos\left(x_i/\sqrt{i}\right) +1
        \end{aligned}\) \vspace*{3ex} \\
    штрафная 1    &\(\left[-50,\,50\right]^D\)     &
        \(\begin{aligned}
            f_{12}(x)= &\textstyle \frac{\pi}{D}\Big\{ 10\,\sin^2(\pi y_1) +\\
            &+\textstyle \sum_{i=1}^{D-1}(y_i-1)^2
                \left[1+10\,\sin^2(\pi y_{i+1})\right] +\\
            &+(y_D-1)^2 \Big\} +\textstyle\sum_{i=1}^D u(x_i,\,10,\,100,\,4)
        \end{aligned}\) \vspace*{2ex} \\
    штрафная 2    &\(\left[-50,\,50\right]^D\)     &
        \(\begin{aligned}
            f_{13}(x)= &\textstyle 0.1 \Big\{\sin^2(3\pi x_1) +\\
            &+\textstyle \sum_{i=1}^{D-1}(x_i-1)^2
                \left[1+\sin^2(3 \pi x_{i+1})\right] + \\
            &+(x_D-1)^2\left[1+\sin^2(2\pi x_D)\right] \Big\} +\\
            &+\textstyle\sum_{i=1}^D u(x_i,\,5,\,100,\,4)
        \end{aligned}\)\\
    сфера         &\(\left[-100,\,100\right]^D\)   &
        \(\begin{aligned}
            \textstyle f_1(x)=\sum_{i=1}^Dx_i^2
        \end{aligned}\) \\
    Schwefel 2.22 &\(\left[-10,\,10\right]^D\)     &
        \(\begin{aligned}
            \textstyle f_2(x)=\sum_{i=1}^D|x_i|+\prod_{i=1}^D|x_i|
        \end{aligned}\) \\
    Schwefel 1.2  &\(\left[-100,\,100\right]^D\)   &
        \(\begin{aligned}
            \textstyle f_3(x)=\sum_{i=1}^D\left(\sum_{j=1}^ix_j\right)^2
        \end{aligned}\) \\
    Schwefel 2.21 &\(\left[-100,\,100\right]^D\)   &
        \(\begin{aligned}
            \textstyle f_4(x)=\max_i\!\left\{\left|x_i\right|\right\}
        \end{aligned}\) \\
    Rosenbrock    &\(\left[-30,\,30\right]^D\)     &
        \(\begin{aligned}
            \textstyle f_5(x)=
            \sum_{i=1}^{D-1}
            \left[100\!\left(x_{i+1}-x_i^2\right)^2+(x_i-1)^2\right]
        \end{aligned}\) \\
    ступенчатая   &\(\left[-100,\,100\right]^D\)   &
        \(\begin{aligned}
            \textstyle f_6(x)=\sum_{i=1}^D\big\lfloor x_i+0.5\big\rfloor^2
        \end{aligned}\) \\
    зашумлённая квартическая &\(\left[-1.28,\,1.28\right]^D\) &
        \(\begin{aligned}
            \textstyle f_7(x)=\sum_{i=1}^Dix_i^4+rand[0,1)
        \end{aligned}\)\vspace*{2ex}\\
    Schwefel 2.26 &\(\left[-500,\,500\right]^D\)   &
        \(\begin{aligned}
        f_8(x)= &\textstyle\sum_{i=1}^D-x_i\,\sin\sqrt{|x_i|}\,+ \\
                &\vphantom{\sum}+ D\cdot
                418.98288727243369
        \end{aligned}\)\\
    Rastrigin     &\(\left[-5.12,\,5.12\right]^D\) &
    \(\begin{aligned}
        \textstyle f_9(x)=\sum_{i=1}^D\left[x_i^2-10\,\cos(2\pi x_i)+10\right]
    \end{aligned}\)\vspace*{2ex}\\
    Ackley        &\(\left[-32,\,32\right]^D\)     &
        \(\begin{aligned}
            f_{10}(x)= &\textstyle -20\, \exp\!\left(
                            -0.2\sqrt{\frac{1}{D}\sum_{i=1}^Dx_i^2} \right)-\\
                       &\textstyle - \exp\left(
                            \frac{1}{D}\sum_{i=1}^D\cos(2\pi x_i)  \right)
                       + 20 + e
        \end{aligned}\) \\
    Griewank      &\(\left[-600,\,600\right]^D\) &
        \(\begin{aligned}
            f_{11}(x)= &\textstyle \frac{1}{4000}\sum_{i=1}^{D}x_i^2 -
                \prod_{i=1}^D\cos\left(x_i/\sqrt{i}\right) +1
        \end{aligned}\) \vspace*{3ex} \\
    штрафная 1    &\(\left[-50,\,50\right]^D\)     &
        \(\begin{aligned}
            f_{12}(x)= &\textstyle \frac{\pi}{D}\Big\{ 10\,\sin^2(\pi y_1) +\\
            &+\textstyle \sum_{i=1}^{D-1}(y_i-1)^2
                \left[1+10\,\sin^2(\pi y_{i+1})\right] +\\
            &+(y_D-1)^2 \Big\} +\textstyle\sum_{i=1}^D u(x_i,\,10,\,100,\,4)
        \end{aligned}\) \vspace*{2ex} \\
    штрафная 2    &\(\left[-50,\,50\right]^D\)     &
        \(\begin{aligned}
            f_{13}(x)= &\textstyle 0.1 \Big\{\sin^2(3\pi x_1) +\\
            &+\textstyle \sum_{i=1}^{D-1}(x_i-1)^2
                \left[1+\sin^2(3 \pi x_{i+1})\right] + \\
            &+(x_D-1)^2\left[1+\sin^2(2\pi x_D)\right] \Big\} +\\
            &+\textstyle\sum_{i=1}^D u(x_i,\,5,\,100,\,4)
        \end{aligned}\)\\
    \midrule%%% тонкий разделитель
    \multicolumn{3}{@{}p{\textwidth}}{%
        \vspace*{-3.5ex}% этим подтягиваем повыше
        \hspace*{2.5em}% абзацный отступ - требование ГОСТ 2.105
        Примечание "---  Для функций \(f_{12}\) и \(f_{13}\)
        используется \(y_i = 1 + \frac{1}{4}(x_i+1)\)
        и~$u(x_i,\,a,\,k,\,m)=
            \begin{cases*}
                k(x_i-a)^m,& \( x_i >a \)\\[-0.5em]
                0,& \( -a\leq x_i \leq a \)\\[-0.5em]
                k(-x_i-a)^m,& \( x_i <-a \)
            \end{cases*}
        $
}\\
\bottomrule %%% нижняя линейка
\end{longtabu}
\endgroup

\section{Форматирование внутри таблиц}\label{app:B3}

В таблице~\ref{tab:other-row} пример с чересстрочным
форматированием. В~файле \verb+userstyles.tex+  задаётся счётчик
\verb+\newcounter{rowcnt}+ который увеличивается на~1 после каждой
строчки (как указано в преамбуле таблицы). Кроме того, задаётся
условный макрос \verb+\altshape+ который выдаёт одно
из~двух типов форматирования в~зависимости от чётности счётчика.

В таблице~\ref{tab:other-row} каждая чётная строчка "--- синяя,
нечётная "--- с наклоном и~слегка поднята вверх. Визуально это приводит
к тому, что среднее значение и~среднеквадратичное изменение
группируются и хорошо выделяются взглядом в~таблице. Сохраняется
возможность отдельные значения в таблице выделить цветом или
шрифтом. К первому и второму столбцу форматирование не применяется
по~сути таблицы, к шестому общее форматирование не~применяется для
наглядности.

Так как заголовок таблицы тоже считается за строчку, то перед ним (для
первого, промежуточного и финального варианта) счётчик обнуляется,
а~в~\verb+\altshape+ для нулевого значения счётчика форматирования
не~применяется.

\begingroup % Ограничиваем область видимости arraystretch
\renewcommand\altshape{
  \ifnumequal{\value{rowcnt}}{0}{
    % Стиль для заголовка таблицы
  }{
    \ifnumodd{\value{rowcnt}}
    {
      \color{blue} % Cтиль для нечётных строк
    }{
      \vspace*{-0.7ex}\itshape} % Стиль для чётных строк
  }
}
\newcolumntype{A}{>{\centering\begingroup\altshape}X[1mc]<{\endgroup}}
\needspace{2\baselineskip}
\renewcommand{\arraystretch}{0.9}%% Уменьшаем  расстояние между
                                %% рядами, чтобы таблица не так много
                                %% места занимала в дисере.
\begin{longtabu} to \textwidth {@{}X[0.27ml]@{}X[0.7mc]@{}A@{}A@{}A@{}X[0.98mc]@{}>{\setlength{\baselineskip}{0.7\baselineskip}}A@{}A<{\stepcounter{rowcnt}}@{}}
% \begin{longtabu} to \textwidth {@{}X[0.2ml]X[1mc]X[1mc]X[1mc]X[1mc]X[1mc]>{\setlength{\baselineskip}{0.7\baselineskip}}X[1mc]X[1mc]@{}}
  \caption{Длинная таблица с примером чересстрочного форматирования\label{tab:other-row}}\vspace*{1ex}\\% label всегда желательно идти после caption
  % \vspace*{1ex}     \\

  \toprule %%% верхняя линейка
\setcounter{rowcnt}{0} &Итера\-ции & JADE\texttt{++} & JADE & jDE & SaDE
& DE/rand /1/bin & PSO \\
 \midrule %%% тонкий разделитель. Отделяет названия столбцов. Обязателен по ГОСТ 2.105 пункт 4.4.5
 \endfirsthead

 \multicolumn{8}{c}{\small\slshape (продолжение)} \\
 \toprule %%% верхняя линейка
\setcounter{rowcnt}{0} &Итера\-ции & JADE\texttt{++} & JADE & jDE & SaDE
& DE/rand /1/bin & PSO \\
 \midrule %%% тонкий разделитель. Отделяет названия столбцов. Обязателен по ГОСТ 2.105 пункт 4.4.5
 \endhead

 \multicolumn{8}{c}{\small\slshape (окончание)} \\
 \toprule %%% верхняя линейка
\setcounter{rowcnt}{0} &Итера\-ции & JADE\texttt{++} & JADE & jDE & SaDE
& DE/rand /1/bin & PSO \\
 \midrule %%% тонкий разделитель. Отделяет названия столбцов. Обязателен по ГОСТ 2.105 пункт 4.4.5
 \endlasthead

 \bottomrule %%% нижняя линейка
 \multicolumn{8}{r}{\small\slshape продолжение следует}     \\
 \endfoot
 \endlastfoot

f1  & 1500 & \textbf{1.8E-60}   & 1.3E-54   & 2.5E-28   & 4.5E-20   & 9.8E-14   & 9.6E-42   \\\nopagebreak
    &      & (8.4E-60) & (9.2E-54) & {\color{red}(3.5E-28)} & (6.9E-20) & (8.4E-14) & (2.7E-41) \\
f2  & 2000 & 1.8E-25   & 3.9E-22   & 1.5E-23   & 1.9E-14   & 1.6E-09   & 9.3E-21   \\\nopagebreak
    &      & (8.8E-25) & (2.7E-21) & (1.0E-23) & (1.1E-14) & (1.1E-09) & (6.3E-20) \\
f3  & 5000 & 5.7E-61   & 6.0E-87   & 5.2E-14   & {\color{green}9.0E-37}   & 6.6E-11   & 2.5E-19   \\\nopagebreak
    &      & (2.7E-60) & (1.9E-86) & (1.1E-13) & (5.4E-36) & (8.8E-11) & (3.9E-19) \\
f4  & 5000 & 8.2E-24   & 4.3E-66   & 1.4E-15   & 7.4E-11   & 4.2E-01   & 4.4E-14   \\\nopagebreak
    &      & (4.0E-23) & (1.2E-65) & (1.0E-15) & (1.8E-10) & (1.1E+00) & (9.3E-14) \\
f5  & 3000 & 8.0E-02   & 3.2E-01   & 1.3E+01   & 2.1E+01   & 2.1E+00   & 2.5E+01   \\\nopagebreak
    &      & (5.6E-01) & (1.1E+00) & (1.4E+01) & (7.8E+00) & (1.5E+00) & (3.2E+01) \\
f6  & 100  & 2.9E+00   & 5.6E+00   & 1.0E+03   & 9.3E+02   & 4.7E+03   & 4.5E+01   \\\nopagebreak
    &      & (1.2E+00) & (1.6E+00) & (2.2E+02) & (1.8E+02) & (1.1E+03) & (2.4E+01) \\
f7  & 3000 & 6.4E-04   & 6.8E-04   & 3.3E-03   & 4.8E-03   & 4.7E-03   & 2.5E-03   \\\nopagebreak
    &      & (2.5E-04) & (2.5E-04) & (8.5E-04) & (1.2E-03) & (1.2E-03) & (1.4E-03) \\
f8  & 1000 & 3.3E-05   & 7.1E+00   & 7.9E-11   & 4.7E+00   & 5.9E+03   & 2.4E+03   \\\nopagebreak
    &      & (2.3E-05) & (2.8E+01) & (1.3E-10) & (3.3E+01) & (1.1E+03) & (6.7E+02) \\
f9  & 1000 & 1.0E-04   & 1.4E-04   & 1.5E-04   & 1.2E-03   & 1.8E+02   & 5.2E+01   \\\nopagebreak
    &      & (6.0E-05) & (6.5E-05) & (2.0E-04) & (6.5E-04) & (1.3E+01) & (1.6E+01) \\
f10 & 500  & 8.2E-10   & 3.0E-09   & 3.5E-04   & 2.7E-03   & 1.1E-01   & 4.6E-01   \\\nopagebreak
    &      & (6.9E-10) & (2.2E-09) & (1.0E-04) & (5.1E-04) & (3.9E-02) & (6.6E-01) \\
f11 & 500  & 9.9E-08   & 2.0E-04   & 1.9E-05   & 7.8E-04  & 2.0E-01   & 1.3E-02   \\\nopagebreak
    &      & (6.0E-07) & (1.4E-03) & (5.8E-05) & (1.2E-03)  & (1.1E-01) & (1.7E-02) \\
f12 & 500  & 4.6E-17   & 3.8E-16   & 1.6E-07   & 1.9E-05   & 1.2E-02   & 1.9E-01   \\\nopagebreak
    &      & (1.9E-16) & (8.3E-16) & (1.5E-07) & (9.2E-06) & (1.0E-02) & (3.9E-01) \\
f13 & 500  & 2.0E-16   & 1.2E-15   & 1.5E-06   & 6.1E-05   & 7.5E-02   & 2.9E-03   \\\nopagebreak
    &      & (6.5E-16) & (2.8E-15) & (9.8E-07) & (2.0E-05) & (3.8E-02) & (4.8E-03) \\
f1  & 1500 & \textbf{1.8E-60}   & 1.3E-54   & 2.5E-28   & 4.5E-20   & 9.8E-14   & 9.6E-42   \\\nopagebreak
    &      & (8.4E-60) & (9.2E-54) & {\color{red}(3.5E-28)} & (6.9E-20) & (8.4E-14) & (2.7E-41) \\
f2  & 2000 & 1.8E-25   & 3.9E-22   & 1.5E-23   & 1.9E-14   & 1.6E-09   & 9.3E-21   \\\nopagebreak
    &      & (8.8E-25) & (2.7E-21) & (1.0E-23) & (1.1E-14) & (1.1E-09) & (6.3E-20) \\
f3  & 5000 & 5.7E-61   & 6.0E-87   & 5.2E-14   & 9.0E-37   & 6.6E-11   & 2.5E-19   \\\nopagebreak
    &      & (2.7E-60) & (1.9E-86) & (1.1E-13) & (5.4E-36) & (8.8E-11) & (3.9E-19) \\
f4  & 5000 & 8.2E-24   & 4.3E-66   & 1.4E-15   & 7.4E-11   & 4.2E-01   & 4.4E-14   \\\nopagebreak
    &      & (4.0E-23) & (1.2E-65) & (1.0E-15) & (1.8E-10) & (1.1E+00) & (9.3E-14) \\
f5  & 3000 & 8.0E-02   & 3.2E-01   & 1.3E+01   & 2.1E+01   & 2.1E+00   & 2.5E+01   \\\nopagebreak
    &      & (5.6E-01) & (1.1E+00) & (1.4E+01) & (7.8E+00) & (1.5E+00) & (3.2E+01) \\
f6  & 100  & 2.9E+00   & 5.6E+00   & 1.0E+03   & 9.3E+02   & 4.7E+03   & 4.5E+01   \\\nopagebreak
    &      & (1.2E+00) & (1.6E+00) & (2.2E+02) & (1.8E+02) & (1.1E+03) & (2.4E+01) \\
f7  & 3000 & 6.4E-04   & 6.8E-04   & 3.3E-03   & 4.8E-03   & 4.7E-03   & 2.5E-03   \\\nopagebreak
    &      & (2.5E-04) & (2.5E-04) & (8.5E-04) & (1.2E-03) & (1.2E-03) & (1.4E-03) \\
f8  & 1000 & 3.3E-05   & 7.1E+00   & 7.9E-11   & 4.7E+00   & 5.9E+03   & 2.4E+03   \\\nopagebreak
    &      & (2.3E-05) & (2.8E+01) & (1.3E-10) & (3.3E+01) & (1.1E+03) & (6.7E+02) \\
f9  & 1000 & 1.0E-04   & 1.4E-04   & 1.5E-04   & 1.2E-03   & 1.8E+02   & 5.2E+01   \\\nopagebreak
    &      & (6.0E-05) & (6.5E-05) & (2.0E-04) & (6.5E-04) & (1.3E+01) & (1.6E+01) \\
f10 & 500  & 8.2E-10   & 3.0E-09   & 3.5E-04   & 2.7E-03   & 1.1E-01   & 4.6E-01   \\\nopagebreak
    &      & (6.9E-10) & (2.2E-09) & (1.0E-04) & (5.1E-04) & (3.9E-02) & (6.6E-01) \\
f11 & 500  & 9.9E-08   & 2.0E-04   & 1.9E-05   & 7.8E-04  & 2.0E-01   & 1.3E-02   \\\nopagebreak
    &      & (6.0E-07) & (1.4E-03) & (5.8E-05) & (1.2E-03)  & (1.1E-01) & (1.7E-02) \\
f12 & 500  & 4.6E-17   & 3.8E-16   & 1.6E-07   & 1.9E-05   & 1.2E-02   & 1.9E-01   \\\nopagebreak
    &      & (1.9E-16) & (8.3E-16) & (1.5E-07) & (9.2E-06) & (1.0E-02) & (3.9E-01) \\
f13 & 500  & 2.0E-16   & 1.2E-15   & 1.5E-06   & 6.1E-05   & 7.5E-02   & 2.9E-03   \\\nopagebreak
    &      & (6.5E-16) & (2.8E-15) & (9.8E-07) & (2.0E-05) & (3.8E-02) & (4.8E-03) \\
\bottomrule %%% нижняя линейка
\end{longtabu} \endgroup

\section{Стандартные префиксы ссылок}\label{app:B4}

Общепринятым является следующий формат ссылок: \texttt{<prefix>:<label>}.
Например, \verb+\label{fig:knuth}+; \verb+\ref{tab:test1}+; \verb+label={lst:external1}+.
В таблице~\ref{tab:tab_pref} приведены стандартные префиксы для различных типов ссылок.

\begingroup
    \centering
    % \small
    \begin{longtable}[c]{|c|c|}
    \caption{Стандартные префиксы ссылок}\label{tab:tab_pref}% label всегда желательно идти после caption
    \\[-0.45\onelineskip]
    \hline
    \textbf{Префикс} & \textbf{Описание} \\ \hline
    \endfirsthead%
    \caption*{\tabcapalign Продолжение таблицы~\thetable}\\[-0.45\onelineskip]
    \hline
    \textbf{Префикс} & \textbf{Описание} \\ \hline
    \endhead
    \hline
    \endfoot
    \hline
    \endlastfoot
    ch:     & Глава             \\
    sec:    & Секция            \\
    subsec: & Подсекция         \\
    fig:    & Рисунок           \\
    tab:    & Таблица           \\
    eq:     & Уравнение         \\
    lst:    & Листинг программы \\
    itm:    & Элемент списка    \\
    alg:    & Алгоритм          \\
    app:    & Секция приложения \\
    \end{longtable}
% \normalsize% возвращаем шрифт к нормальному
\endgroup

Для упорядочивания ссылок можно использовать разделительные символы.
Например, \verb+\label{fig:scheemes/my_scheeme}+ или \\ \verb+\label{lst:dts/linked_list}+.

\section{Очередной подраздел приложения}~\label{app:B5}

Нужно больше подразделов приложения!

\section{И ещё один подраздел приложения}~\label{app:B6}

Нужно больше подразделов приложения!
        % Приложения
\chapter*{Резюме}
\addcontentsline{toc}{chapter}{Резюме}
	Имя:

	Дата рождения:

	Место рождения:

	Гражданство:

Контактные данные:

	E-mail:

Образование:

201 (год) – 201 (год) Университет ИТМО – к.т.н.

201 (год) – 201 (год) Университет ИТМО Магистр

201 (год) – 201 (год) Университет ИТМО Бакалавр

Владение языками

Английский Свободно

Профессиональное трудоустройство

201 (год) –

\end{document}
